\section{Introduction.}

\par Currently, deep learning is an open-ended research area. With their success, deep learning is required to constantly increase the ability to go around data and relatively cheap graphics processors, allowing to build an effective calculation procedure.
In-depth training through sequential non-linear transformations, which, as a rule, are represented in the form of artificial neural networks. Today, they are used to solve such problems as prediction, pattern recognition and a number of others.
\par Deep learning is a subset of machine learning methods that use artificial neuron networks, built on the basis of an analogy with the structure of the neurons of the human brain. The term “deep” implies a large number of layers in the neural network.
\par Google, Microsoft, Amazon, Apple, Facebook and many other deep learning methods are used to analyze large data sets. Now this knowledge and skills went beyond purely academic research and became the property of large industrial companies.
\par The relevance of the topic of deep learning is confirmed by the regular appearance of articles on this topic.

\par The aim of the thesis is the development of teaching tools for the study of models of deep learning. In accordance with the purpose of the study, the following tasks were set:
\begin{itemize}
    \item analyze the main laboratory work;
    \item compare and choose technology for modeling deep neyron
networks;
    \item develop and test several laboratory works.
    \end{itemize}