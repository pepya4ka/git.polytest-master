\section{Conclusion.}

As a result of the bachelor's work, a study was made of deep learning technologies: the existing and new technologies that were applied, as well as various deep learning programs were reviewed. The analysis of laboratory work laid out in open access was made.
TensorFlow and Keras libraries were chosen among the many software tools for deep learning. They meet the needs of the course being developed, namely, it is suitable for solving assigned tasks.
To visualize the search results, Jupyter Notebook graphic web shell was selected. The design was carried out in a graphic web shell Jupyter Notebook. In it were the laboratory works themselves.

The practical significance of the development of an educational-methodical complex of laboratory works lies in the possibility of introducing it into the educational process.
Performing a course of laboratory work on the subject “study of models of deep learning” consists of 3 works. Each laboratory focuses on the study of various tasks: classical fictions, image localization and time series prediction. Upon their completion, students will learn to reduce various tasks to formal statements of machine learning problems and to apply the necessary technologies and schemes for their solutions.
The tasks of the bachelor's work were fulfilled, and the goals were achieved.