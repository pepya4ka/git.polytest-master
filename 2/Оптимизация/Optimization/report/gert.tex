\newpage
\section{Решение задачи оценки показателей эффективности стохастической сети с использованием методики GERT. Выбор и использование математического пакета Matlab для решения сформулированной задачи.}

\subsection{Постановка задачи}
Дано:
\begin{enumerate}
	\item {Граф GERT-сети (рисунок 1).

	\begin{figure}[!h]
	\centering
	\begin{tikzpicture}[->,>=stealth',shorten >=1pt,auto,node distance=4cm,
	  thick,main node/.style={circle,fill=blue!20,draw,font=\sffamily\Large\bfseries}]

	  \node[main node] (1) {1};
	  \node[main node] (2) [above right of=1] {2};
	  \node[main node] (3) [below of=1] {3};
	  \node[main node] (4) [below of=2] {4};
	  \node[main node] (5) [below right of=2] {5};
	  \node[main node] (6) [below of=5] {6};

	  \path[every node/.style={font=\sffamily\small}]
		(1)	edge node [left] {0,5} (2)
			edge node [left] {0,5} (4)
		(2)	edge node [left] {} (5)
		(3)	edge node [left] {0,6} (1)
			edge [loop below] node {0,4} (3)
		(4)	edge node [left] {0,1} (2)
			edge node [left] {0,3} (3)
			edge node [left] {0,2} (5)
			edge node [left] {0,4} (6)
		(5)	edge node [left] {} (6);
	\end{tikzpicture}
	\caption{Граф GERT-сети.}
	\end{figure}
}	
	
	\item Каждой дуге-работе $(i,j)$ поставлены в соответствие следующие данные:
	\begin{enumerate}
		\item Закон распределения времени выполнения работы. Будем считать его нормальным.
		\item Параметры закона распределения (математическое ожидание $M$ и дисперсия $D$).
		\item Вероятность $P_{ij}$ выполнения работы, показанная на графе.
	\end{enumerate}
\end{enumerate}

\begin{table}[htb]
\centering
	\begin{tabular}{|c|c|c|c|}
	\hline 
	Начальная вершина & Конечная вершина & $M$ & $D$ \\ 
	\hline 
	1 & 2 & 12 & 9 \\ 
	\hline 
	1 & 4 & 28 & 16 \\ 
	\hline 
	2 & 5 & 14 & 9 \\ 
	\hline 
	3 & 1 & 11 & 4 \\ 
	\hline 
	3 & 3 & 33 & 16 \\ 
	\hline 
	4 & 2 & 11 & 4 \\ 
	\hline 
	4 & 3 & 33 & 25 \\ 
	\hline 
	4 & 5 & 45 & 25 \\ 
	\hline 
	4 & 6 & 23 & 16 \\ 
	\hline 
	5 & 6 & 43 & 25 \\ 
	\hline 
	\end{tabular} 
\caption{Параметры закона распределения для дуг графа}
\end{table}

Найти:
\begin{enumerate}
	\item Вероятность выхода в завершающий узел графа (для всех вариантов узел 6).
	\item Математическое ожидание.
	\item Дисперсию времени выхода процесса в завершающий узел графа.
\end{enumerate}

Перечислить все петли всех порядков, обнаруженные на графе, выписать уравнение Мейсона, получить решение для $W_E(s)$ и найти требуемые параметры. Примерно так, как это сделано в примере на стр. 403 -- 409 книги Филипса и Гарсиа «Методы анализа сетей»

\subsection{Ход работы}
Решение:

Замкнём граф дугой из вершины 6 в вершину 1 (рисунок 2).

\begin{figure}[!h]
	\centering
	\begin{tikzpicture}[->,>=stealth',shorten >=1pt,auto,node distance=4cm,
	  thick,main node/.style={circle,fill=blue!20,draw,font=\sffamily\Large\bfseries}]

	  \node[main node] (1) {1};
	  \node[main node] (2) [above right of=1] {2};
	  \node[main node] (3) [below of=1] {3};
	  \node[main node] (4) [below of=2] {4};
	  \node[main node] (5) [below right of=2] {5};
	  \node[main node] (6) [below of=5] {6};

	  \path[every node/.style={font=\sffamily\small}]
		(1)	edge node [left] {0,5} (2)
			edge node [left] {0,5} (4)
		(2)	edge node [left] {} (5)
		(3)	edge node [left] {0,6} (1)
			edge [loop below] node {0,4} (3)
		(4)	edge node [left] {0,1} (2)
			edge node [left] {0,3} (3)
			edge node [left] {0,2} (5)
			edge node [left] {0,4} (6)
		(5)	edge node [left] {} (6)
		(6)	edge [red, bend left] node [left] {$\frac{1}{W_E}$} (1);
	\end{tikzpicture}
	\caption{Замкнутый граф GERT-сети.}
\end{figure}

Петли первого порядка:
\begin{itemize}
\item $W_{12}W_{25}W_{56} \frac{1}{W_E}$
\item $W_{14}W_{42}W_{25}W_{56} \frac{1}{W_E}$
\item $W_{14}W_{43}W_{31}$
\item $W_{14}W_{45}W_{56} \frac{1}{W_E}$
\item $W_{14}W_{46} \frac{1}{W_E}$
\item $W_{33}$
\end{itemize}

Петли второго порядка:
\begin{itemize}
\item $W_{33}$ и $W_{12}W_{25}W_{56} \frac{1}{W_E}$
\item $W_{33}$ и $W_{14}W_{42}W_{25}W_{56} \frac{1}{W_E}$
\item $W_{33}$ и $W_{14}W_{45}W_{56} \frac{1}{W_E}$
\item $W_{33}$ и $W_{14}W_{46} \frac{1}{W_E}$
\end{itemize}

Петель третьего порядка нет.

Выпишем уравнение Мейсона:
\begin{eqnarray*}
	H = 1
		& - & W_{12}W_{25}W_{56} \frac{1}{W_E} \\
		& - & W_{14}W_{42}W_{25}W_{56} \frac{1}{W_E} \\
		& - & W_{14}W_{43}W_{31} \\
		& - & W_{14}W_{45}W_{56} \frac{1}{W_E} \\
		& - & W_{14}W_{46} \frac{1}{W_E} \\
		& - & W_{33} \\
		& + & W_{33}W_{12}W_{25}W_{56} \frac{1}{W_E} \\
		& + & W_{33}W_{14}W_{42}W_{25}W_{56} \frac{1}{W_E} \\
		& + & W_{33}W_{14}W_{45}W_{56} \frac{1}{W_E} \\
		& + & W_{33}W_{14}W_{46} \frac{1}{W_E} = 0
\end{eqnarray*}

Отсюда выведем $W_E(S)$:
\begin{eqnarray*}
	1 - W_{14}W_{43}W_{31} - W_{33} = && (W_{12}W_{25}W_{56} + W_{14}W_{42}W_{25}W_{56} + \\
	&& W_{14}W_{45}W_{56} + W_{14}W_{46} - W_{33}W_{12}W_{25}W_{56} - \\
	&& W_{33}W_{14}W_{42}W_{25}W_{56} - W_{33}W_{14}W_{45}W_{56} - W_{33}W_{14}W_{46})\frac{1}{W_E}
\end{eqnarray*}

\begin{eqnarray*}
	W_E(S) = && (W_{12}W_{25}W_{56} + W_{14}W_{42}W_{25}W_{56} + W_{14}W_{45}W_{56} + W_{14}W_{46} - \\
	&&W_{33}W_{12}W_{25}W_{56} - W_{33}W_{14}W_{42}W_{25}W_{56} - W_{33}W_{14}W_{45}W_{56} - W_{33}W_{14}W_{46})/ \\
	&&(1 - W_{14}W_{43}W_{31} - W_{33})
\end{eqnarray*}

Далее рассчитаем W-функции дуг.

\begin{table}[htb]
\centering
	\begin{tabular}{|c|c|c|c|c|c|}
	\hline 
	Начальная вершина & Конечная вершина & Вес ребра ($p_{ij}$) & $M$ & $D$ & W-функция \\ 
	\hline 
	1 & 2 & 0,5 & 12 & 9 & $0,5*exp(12s+\frac{9}{2}s^2)$ \\ 
	\hline 
	1 & 4 & 0,5 & 28 & 16 & $0,5*exp(28s+\frac{16}{2}s^2)$ \\ 
	\hline 
	2 & 5 & 1 & 14 & 9 & $exp(14s+\frac{9}{2}s^2)$ \\ 
	\hline 
	3 & 1 & 0,6 & 11 & 4 & $0,6*exp(11s+\frac{4}{2}s^2)$ \\ 
	\hline 
	3 & 3 & 0,4 & 33 & 16 & $0,4*exp(33s+\frac{16}{2}s^2)$ \\ 
	\hline 
	4 & 2 & 0,1 & 11 & 4 & $0,1*exp(11s+\frac{4}{2}s^2)$ \\ 
	\hline 
	4 & 3 & 0,3 & 33 & 25 & $0,3*exp(33s+\frac{25}{2}s^2)$ \\ 
	\hline 
	4 & 5 & 0,2 & 45 & 25 & $0,2*exp(45s+\frac{25}{2}s^2)$ \\ 
	\hline 
	4 & 6 & 0,4 & 23 & 16 & $0,4*exp(23s+\frac{16}{2}s^2)$ \\ 
	\hline 
	5 & 6 & 1 & 43 & 25 & $exp(43s+\frac{25}{2}s^2)$ \\ 
	\hline 
	\end{tabular} 
\caption{Производящие функции моментов}
\end{table}

Далее вычислим математическое ожидание и дисперсию: $M_E(s) = 1$ при $s=0$

Поскольку $W_E(s)=p_E M_E (s)$,  то  $p_E=W_E(0)$,  откуда следует, что $M_E(s)=\frac{W_E(s)}{p_E} =\frac{W_E(s)}{W_E(0)}$

Вычисляя первую и вторую производные по $s$ функции $M_E(s)$, и полагая $s=0$, находим математическое ожидание:
$\mu_{1E}=\frac{\partial M_E(s)}{\partial s}|s=0$

и дисперсию:

$\sigma^2=\mu_{2E}-[\mu_{1E}]^2$.

Вероятность выхода в завершающий узел графа:

$p_E=W_E (0)$.

Для моделирования работы был написан скрипт, представленный в листинге 7.

\lstinputlisting[language=Matlab, caption={Код для вычисления заданных выражений}]{../task2/main.m}

\newpage

Результаты работы представлены в листинге 8.

\lstinputlisting[language={},caption={Результат работы скрипта}]{../task2/main.out}

\subsection{Результат}

По итогам проведённых расчётов были получены следующие результаты:
\begin{enumerate}
	\item Вероятность выхода в завершающий узел графа равна 100\% ($p=W_E=1$).
	\item Математическое ожидание 88,46.
	\item Дисперсия времени выхода процесса в завершающий узел графа 2472,3.
\end{enumerate}
