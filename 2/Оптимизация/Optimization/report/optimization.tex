\newpage
\section{Варианты формализации многокритериальной задачи и их решение с использованием Optimization Toolbox  системы Matlab.}


\subsection{Постановка задачи}
Мебельная  фабрика выпускает столы, стулья, бюро и книжные шкафы. При изготовлении используются два типа досок, причем фабрика имеет в наличии 1500 м досок первого типа и 1000 м досок второго типа. Кроме того, заданы трудовые ресурсы в количестве 800 чел/час. В таблице приводятся нормативы затрат каждого из видов ресурсов на изготовление 1 ед изделия и прибыль от реализации 1  ед  изделия.

\begin{table}[htb]
	\begin{tabularx}{\textwidth}{|X|c|c|c|c|}
	\hline 
	\multirow{2}{*}{Ресурсы} & \multicolumn{4}{c|}{Затраты на 1 ед изделия} \\ 
	\hhline{~----}
	{} & столы & стулья & бюро & Книжные шкафы \\ 
	\hline 
	Доски первого типа, м & 5 & 1 & 9 & 12 \\ 
	\hline 
	Доски второго типа, м & 2 & 3 & 4 & 1 \\ 
	\hline 
	Трудовые ресурсы, чел/час & 3 & 2 & 5 & 10 \\ 
	\hline 
	Прибыль, руб/шт & 12 & 5 & 15 & 10 \\ 
	\hline 
	\end{tabularx} 
\caption{Нормативы затрат ресурсов на единицу изделия}
\end{table}

По этим исходным данным решить задачу определения оптимальный ассортимент, максимизирующий прибыль (разность между выручкой и расходами.) и выручку при следующих ценах изготавливаемую мебель:

\begin{itemize}
\item стол -- 32 руб;
\item стул -- 15 руб;
\item бюро -- 12 руб;
\item книжный шкаф -- 80 руб.
\end{itemize}

В отчёте необходимо описать:
\begin{enumerate}
	\item Осуществление перехода от многокритериальной задачи к однокритериальной с использованием различных подходов.
	\item Решение задачи стохастического программирования для одной из однокритериальных задач, превратив детерминированное ограничение в вероятностное по схеме:
	$P(\sum\limits_{j=1}^n a_{ij}k_j-b_j\leq0)\geq\alpha_i$
	
	Менять $\alpha_i$ в следующем диапазоне $0.1 \leq \alpha_i \leq 0.9$.
	
	Считать случайной величиной $b_i$ или элементы $\{a_{ij}\}$ $i$-й строки матрицы $A$ $\{a_{ij}\}$ (по выбору).
\end{enumerate}

\subsection{Выделение главного критерия}
Выбирается один из критериев, например $C_i$, который наиболее полно отражает цель принятия решений. Остальные критерии учитываются только с точки зрения возможного указания их нижних границ $C_j(a) \geq \gamma_i$, $ j\neq i$. Таким образом, исходная задача многокритериального принятия решений заменяется однокритериальной задачей с критерием $C_i$, т.е. $a^* = \text{arg max } C_i(a)$, при ограничениях $C_k (a) \geq \gamma_i$, $k\neq i$.

Критерии:
\begin{itemize}
\item $max (12x_1+5x_2+15x_3+10x_4)$ (прибыль)
\item $max (32x_1+15x_2+12x_3+80x_4)$ (выручка)
\end{itemize}

Ограничения:
\begin{itemize}
\item $5x_1+x_2+9x_3+12x_4 \leq 1500$ (доски первого типа)
\item $2x_1+3x_2+4x_3+1x_4 \leq 1000$ (доски второго типа)
\item $3x_1+2x_2+5x_3+1x_4 \leq 800$ (трудовые ресурсы)
\end{itemize}

\subsubsection{Максимизация выручки}

Целевая функция:

$f = min (-32x_1-15x_2-12x_3-80x_4)$

Начальные условия:

$x_0 =
\begin{pmatrix}
  0 \\
  0 \\
  0 \\
  0
\end{pmatrix}$

Ограничения:

$A =
\begin{pmatrix}
  5 & 1 & 9 & 12 \\
  2 & 3 & 4 & 1 \\
  3 & 2 & 5 & 1 \\
  -1& 0 & 0 & 0 \\
  0 &-1 & 0 & 0 \\
  0 & 0 &-1 & 0 \\
  0 & 0 & 0 & -1
\end{pmatrix}$

$b =
\begin{pmatrix}
  1500 \\
  1000 \\
  800 \\
  0 \\
  0 \\
  0 \\
  0
\end{pmatrix}$

\lstinputlisting[language=Matlab, caption={Поиск оптимального решения для максимизация выручки}]{../task1/max_gain.m}

Результат:
\begin{itemize}
\item $x_1 = -0,0000$
\item $x_2 = 300,0000$
\item $x_3 = 0$
\item $x_4 = 100,0000$
\item $f1 = -2500$
\item $f2 = -12500$
\end{itemize}

\subsubsection{Максимизация прибыли}

Целевая функция:

$f = min (-12x_1-5x_2-15x_3-10x_4)$

Начальные условия:

$x_0 =
\begin{pmatrix}
  0 \\
  0 \\
  0 \\
  0
\end{pmatrix}$

Ограничения:

$A =
\begin{pmatrix}
  5 & 1 & 9 & 12 \\
  2 & 3 & 4 & 1 \\
  3 & 2 & 5 & 1 \\
  -1& 0 & 0 & 0 \\
  0 &-1 & 0 & 0 \\
  0 & 0 &-1 & 0 \\
  0 & 0 & 0 & -1 \\
  -32 & -15 & -12 & -80
\end{pmatrix}$

$b =
\begin{pmatrix}
  1500 \\
  1000 \\
  800 \\
  0 \\
  0 \\
  0 \\
  0 \\
  -12500
\end{pmatrix}$

\newpage
\lstinputlisting[language=Matlab, caption={Поиск оптимального решения для максимизация прибыли}]{../task1/max_profit.m}

Результат:
\begin{itemize}
\item $x_1 = 261,2903$
\item $x_2 = 0$
\item $x_3 = 0,0000$
\item $x_4 = 16,1290$
\item $f1 = -3296,8$
\item $f2 = -9651,6$
\end{itemize}

\subsection{Свертка критериев}

Максимизируется критерий объединенной операции, получающийся в результате суммирования всех частных критериев:

$C(a)=\sum\limits_{i=1}^m w_i C_i^n (a)$

$C_i^n (a)=\frac{C_i (a)}{C_i^*}$

$C_i^*$ - оптимальное решение задачи по каждому критерию в отдельности, $w_1+w_2+\dots+w_m=1$.

\lstinputlisting[language=Matlab, caption={Свертка критериев}]{../task1/convolution.m}

В fmincon передается сумма нормированных значений (первый критерий делится на f1, второй на f2), каждое из которых умножено на определенный весовой коэффициент. Результат:
\begin{itemize}
\item $x_1 = 166,4573$
\item $x_2 = 127,8185$
\item $x_3 = -0,0000$
\item $x_4 = 44,9913$
\item $f = -0,9019$ (суммарное)
\end{itemize}

\subsection{Максимин или минимакс}

Максиминную свертку представим в следующем виде: $C_i(a)= \text{min } w_i C_i(a)$

Решение $a^*$ является наилучшим, если для всех $a$ выполняется условие $C(a^*) \geq C(a)$, или $a^* = \text{arg max } C(a) = \text{arg max min } w_i C_i (a)$.

Решение задачи представлено как программа в среде Matlab, с использованием функции fminimax:

$f_1=((12x_1+5x_2+15x_3+10x_4)/3214)^{-1}$;

$f_2=((32x_1+15x_2+12x_3+80x_4 )^2/12500)^{-1}$;

\lstinputlisting[language=Matlab, caption={Содержание файла maxmin.m}]{../task1/maxmin.m}

\newpage
\lstinputlisting[language=Matlab, caption={Содержание файла funminmax.m}]{../task1/funminmax.m}

Так как в среде Matlab реализована только функция fminimax, которая минимизирует наихудшие значения системы функций нескольких переменных, начиная со стартовой оценки ($x_0$), то для реализации максиминной свертки необходимо в fminimax передавать функции, возведенные в степень "-1" (функция funminmax).

Результат:
\begin{itemize}
\item $x_1 = 111,6707$
\item $x_2 = 201,6612$
\item $x_3 = -0,0000$
\item $x_4 = 61,6654$
\item $f1 = 1,0840$
\item $f2 = 1,0840$
\end{itemize}

\subsection{Метод последовательных уступок}

Для решения данной задачи была выбрана уступка = 10\%. Решение задачи представлено как программа в среде Matlab, с использованием функции fmincon.

Целевые функции:
\begin{itemize}
\item $f_1=-(12x_1+5x_2+15x_3+10x_4)$
\item $f_2=-(32x_1+15x_2+12x_3+80x_4)$
\end{itemize}

Для первого критерия:

$A =
\begin{pmatrix}
  5 & 1 & 9 & 12 \\
  2 & 3 & 4 & 1 \\
  3 & 2 & 5 & 1 \\
  -1& 0 & 0 & 0 \\
  0 &-1 & 0 & 0 \\
  0 & 0 &-1 & 0 \\
  0 & 0 & 0 & -1
\end{pmatrix}$

$b =
\begin{pmatrix}
  1500 \\
  1000 \\
  800 \\
  0 \\
  0 \\
  0 \\
  0
\end{pmatrix}$

Результат:
\begin{itemize}
\item $x_1 = 261,29$
\item $x_2 = 0$
\item $x_3 = 0$
\item $x_4 = 16,13$
\item $f1 = 3297$
\item $f2 = 9651$
\end{itemize}

3297 – 329,7 = 2967,3  (10\%)

Для второго критерия:

$A =
\begin{pmatrix}
  5 & 1 & 9 & 12 \\
  2 & 3 & 4 & 1 \\
  3 & 2 & 5 & 1 \\
  -1& 0 & 0 & 0 \\
  0 &-1 & 0 & 0 \\
  0 & 0 &-1 & 0 \\
  0 & 0 & 0 & -1 \\
  -5 & -1 & -9 & -12
\end{pmatrix}$

$b =
\begin{pmatrix}
  1500 \\
  1000 \\
  800 \\
  0 \\
  0 \\
  0 \\
  0 \\
  -2967,3
\end{pmatrix}$

Результат:
\begin{itemize}
\item $x_1 = 112,7$
\item $x_2 = 200,3$
\item $x_3 = 0$
\item $x_4 = 61,4$
\item $f1 = 2967$
\item $f2 = 11519$
\end{itemize}

\subsection{Fgoalattain}

fgoalattain решает задачу достижения цели, которая является одной из формулировок задач для векторной оптимизации.

x = fgoalattain(fun, $x_0$, goal, weight):
\begin{itemize}
\item fun -- целевая функция, 
\item $х_0$ -- начальные значения,
\item goal -- целевые значения,
\item weight -- веса.
\end{itemize}

Решение задачи представлено как программа в среде Matlab, с использованием функций fminicon и fgoalattain.

Целевые значения:

Goal =(15855000 10240038400036 68000000 38080000 4900000) 

\newpage
Веса:

weight=abs(goal) -- для того, чтобы приближение к критериям было одинаково

\lstinputlisting[language=Matlab, caption={Содержание файла fgoalattain.m}]{../task1/fgoalattain.m}

Результат:
\begin{itemize}
\item $x_1 = 207,81$
\item $x_2 = 72,08$
\item $x_3 = 0$
\item $x_4 = 32,4$
\item $f1 = 3178$
\item $f2 = 10324$
\item $Att. = 0,1741$
\end{itemize}

\subsection{Задача стохастического программирования}

Требуется найти такие $x_1$, $x_2$, $x_3$, $x_4$ для которых выполняться следующие ограничения:
\begin{itemize}
\item $5x_1+x_2+9x_3+12x_4 \leq 1500$
\item $2x_1+3x_2+4x_3+1x_4 \leq 1000$
\item $3x_1+2x_2+5x_3+1x_4 \leq 800$
\end{itemize}

Перейдем от последнего ограничения к вероятностному по схеме:
$P(\sum\limits_{j=1}^n a_{ij}k_j-b_j\leq0)\geq\alpha_i$
	
И будем менять $\alpha_i$ в диапазоне $0.1 \leq \alpha_i \leq 0.9$, и возьмём коэффициенты $a_i$ за случайные величины.

$P(0,6x_1 + 0,8x_2 + 1,0x_3 + 1,2x_4 \leq 100) \geq \alpha_i$

Пользуясь формулой:
$\sum\limits_{j=1}^3 a_{ij} x_j - b + K_\alpha \sigma_A \leq 0$

получим вероятностное ограничение для задачи, где  a = {0,6, 0,8, 1,0, 1,2}, b = 100 -- взяты из первоначального вида ограничения, $\sigma_A = \sqrt{x \text{ cov(a) } x^T}$.

По таблице функции распределения стандартного нормального закона находим коэффициенты $K_\alpha (0,5 \leq  \alpha \leq  0,9)$:
\begin{itemize}
\item $K_{0,5} = 0$
\item $K_{0,6} = 0,253$
\item $K_{0,7} = 0,520$
\item $K_{0,8} = 0,841$
\item $K_{0,9} = 1,282$
\end{itemize}

\lstinputlisting[language=Matlab]{../task1/st1.m}
\lstinputlisting[language=Matlab]{../task1/st2.m}
\lstinputlisting[language=Matlab]{../task1/st3.m}

\begin{table}[htb]
	\begin{tabularx}{\textwidth}{|X|X|X|X|X|X|}
	\hline 
	\multirow{2}{*}{} & \multicolumn{5}{c|}{K} \\ 
	\hhline{~-----}
	{} & 0 & 0,253 & 0,52 & 0,841 & 1,282 \\ 
	\hline 
	$x_1$ & 100 & 29,2780 & 27,7214 & 30,0887 & 25,4234 \\ 
	\hline 
	$x_2$ & 0 & 28,7760 & 25,8883 & 26,9406 & 22,6038 \\ 
	\hline 
	$x_3$ & 40 & 28,4840 & 24,8369 & 25,4863 & 21,2327 \\ 
	\hline 
	$x_4$ & 0 & 15,4300 & 12,8728 & 0,5164 & 1,1430 \\ 
	\hline 
	$f$ & 1800 & 1076,8 & 963,4 & 883,2 & 748 \\
	\hline 
	\end{tabularx} 
\caption{Результаты}
\end{table}

Видно, что задача чувствительна к выбранному ограничению, т.к. для различных K получились разные результаты. Так же следует отметить, что значения функций соответствуют нормальному закону распределения, что соответствует теории.