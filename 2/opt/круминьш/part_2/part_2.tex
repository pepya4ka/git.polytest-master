\documentclass[a4paper, 12pt]{report}		% general format
\usepackage{multicol}
%%%% Charset
\usepackage{cmap}							% make PDF files searchable and copyable
\usepackage{bm}
\usepackage{pdfpages}
\usepackage[utf8x]{inputenc} 				% accept different input encodings
\usepackage[english,russian]{babel}   %% загружает пакет многоязыковой вёрстки
%\usepackage{fontspec}      %% подготавливает загрузку шрифтов Open Type, True Type и др.
%\defaultfontfeatures{Ligatures={TeX},Renderer=Basic}  %% свойства шрифтов по умолчанию
%\setmainfont[Ligatures={TeX,Historic}]{Roboto-Light} %% задаёт основной шрифт документа
%\setsansfont{Roboto-Light}  
\usepackage{float}
%%%% Graphics
%\usepackage[dvipsnames]{xcolor}			% driver-independent color extensions
\usepackage{graphicx}						% enhanced support for graphics
\usepackage{wrapfig}						% produces figures which text can flow around

%%%% Math
\usepackage{amsmath}						% American Mathematical Society (AMS) math facilities
\usepackage{amsfonts}						% fonts from the AMS
\usepackage{amssymb}						% additional math symbols

%%%% Typograpy (don't forget about cm-super)
\usepackage{microtype}						% subliminal refinements towards typographical perfection
\linespread{1.0}							% line spacing
\usepackage[mag=1000, left=2.0cm, right=1.5cm, top=2cm, bottom=2cm, headsep=0.7cm, footskip=1cm]{geometry}
\setlength{\parindent}{0pt}					% we don't want any paragraph indentation
\usepackage{parskip}						% some distance between paragraphs

%%%% Tables
\usepackage{tabularx}						% tables with variable width columns
\usepackage{multirow}						% for tabularx
\usepackage{hhline}							% for tabularx
\usepackage{tabu}
\usepackage{longtable}

%%%% Graph
\usepackage{tikz}							% package for creating graphics programmatically
\usetikzlibrary{arrows}						% edges for tikz

%%%% Other
\usepackage{url}							% verbatim with URL-sensitive line breaks
\usepackage{fancyvrb}						% sophisticated verbatim text (with box)

\usepackage{fancyhdr}
\usepackage{latexsym}
\usepackage{booktabs}
\usepackage{array}

\usepackage{listings}
\usepackage{caption}
\DeclareCaptionFont{white}{\color{white}}
\DeclareCaptionFormat{listing}{\colorbox{gray}{\parbox{\dimexpr\textwidth-1.72\fboxsep\relax}{#1#2#3}}}
\captionsetup[lstlisting]{format=listing,labelfont=white,textfont=white,margin=0pt}
\lstset{language=C,
	basicstyle=\footnotesize,
	keepspaces=true,
	tabsize=4,               
	frame=single,                           % Single frame around code
	rulecolor=\color{black},
	captionpos=b,
	showstringspaces=false,	
	abovecaptionskip=-0.9pt,
	xleftmargin=3.4pt,
	xrightmargin=2.6pt,
	breaklines=true,
	postbreak=\raisebox{0ex}[0ex][0ex]{\ensuremath{\color{black}\hookrightarrow\space}},
	xleftmargin=3.2pt,
	literate={а}{{\selectfont\char224}}1
	{~}{{\textasciitilde}}1
	{б}{{\selectfont\char225}}1
	{в}{{\selectfont\char226}}1
	{г}{{\selectfont\char227}}1
	{д}{{\selectfont\char228}}1
	{е}{{\selectfont\char229}}1
	{ё}{{\"e}}1
	{ж}{{\selectfont\char230}}1
	{з}{{\selectfont\char231}}1
	{и}{{\selectfont\char232}}1
	{й}{{\selectfont\char233}}1
	{к}{{\selectfont\char234}}1
	{л}{{\selectfont\char235}}1
	{м}{{\selectfont\char236}}1
	{н}{{\selectfont\char237}}1
	{о}{{\selectfont\char238}}1
	{п}{{\selectfont\char239}}1
	{р}{{\selectfont\char240}}1
	{с}{{\selectfont\char241}}1
	{т}{{\selectfont\char242}}1
	{у}{{\selectfont\char243}}1
	{ф}{{\selectfont\char244}}1
	{х}{{\selectfont\char245}}1
	{ц}{{\selectfont\char246}}1
	{ч}{{\selectfont\char247}}1
	{ш}{{\selectfont\char248}}1
	{щ}{{\selectfont\char249}}1
	{ъ}{{\selectfont\char250}}1
	{ы}{{\selectfont\char251}}1
	{ь}{{\selectfont\char252}}1
	{э}{{\selectfont\char253}}1
	{ю}{{\selectfont\char254}}1
	{я}{{\selectfont\char255}}1
	{А}{{\selectfont\char192}}1
	{Б}{{\selectfont\char193}}1
	{В}{{\selectfont\char194}}1
	{Г}{{\selectfont\char195}}1
	{Д}{{\selectfont\char196}}1
	{Е}{{\selectfont\char197}}1
	{Ё}{{\"E}}1
	{Ж}{{\selectfont\char198}}1
	{З}{{\selectfont\char199}}1
	{И}{{\selectfont\char200}}1
	{Й}{{\selectfont\char201}}1
	{К}{{\selectfont\char202}}1
	{Л}{{\selectfont\char203}}1
	{М}{{\selectfont\char204}}1
	{Н}{{\selectfont\char205}}1
	{О}{{\selectfont\char206}}1
	{П}{{\selectfont\char207}}1
	{Р}{{\selectfont\char208}}1
	{С}{{\selectfont\char209}}1
	{Т}{{\selectfont\char210}}1
	{У}{{\selectfont\char211}}1
	{Ф}{{\selectfont\char212}}1
	{Х}{{\selectfont\char213}}1
	{Ц}{{\selectfont\char214}}1
	{Ч}{{\selectfont\char215}}1
	{Ш}{{\selectfont\char216}}1
	{Щ}{{\selectfont\char217}}1
	{Ъ}{{\selectfont\char218}}1
	{Ы}{{\selectfont\char219}}1
	{Ь}{{\selectfont\char220}}1
	{Э}{{\selectfont\char221}}1
	{Ю}{{\selectfont\char222}}1
	{Я}{{\selectfont\char223}}1,
	extendedchars=true
}

%галочка
\usepackage{amssymb}% http://ctan.org/pkg/amssymb
\usepackage{pifont}% http://ctan.org/pkg/pifont
\newcommand{\cmark}{\ding{52}}%
\newcommand{\xmark}{\ding{56}}
%------------------------------------------------------------------------------
\renewcommand{\labelenumii}{\theenumii}
\renewcommand{\theenumii}{\theenumi.\arabic{enumii}.}
\addto\captionsrussian{\def\refname{Список использованных источников}}
\begin{document}
\begin{titlepage}
\thispagestyle{empty}

\begin{center}
Санкт-Петербургский политехнический университет Петра Великого\\
Институт Информационных Технологий и Управления\\*
Кафедра компьютерных систем и программных технологий\\*
\hrulefill
\end{center}

\vspace{15em}

\begin{center}
\textsc{\textbf{Курсовая работа}}
\vspace{1em}

Дисциплина: \textbf{Методы оптимизации}
\vspace{2em}

Тема: \textbf{Формулировка и решение задачи выбора оптимального решения с использованием различных математических моделей}
\end{center}

\vspace{16em}

\begin{flushleft}
Выполнил студент гр. 53501/3 \hrulefill С.А. Мартынов \\
\vspace{1.5em}
Руководитель, к.т.н.,доц. \hrulefill А.Г. Сиднев\\
\end{flushleft}

\vspace{\fill}

\begin{center}
Санкт-Петербург \\
2015
\end{center}

\end{titlepage}
\setcounter{page}{2}
\tableofcontents
\clearpage

%------------------------------------------------------------------------------
%\input{intro}
\setcounter{chapter}{1}
\chapter{Марковские модели принятия решений}
\section{Постановка задачи}
\textbf{Вариант:} 12, решить задачу методом итераций по стратегиям для $N=\infty$\\\\
%\textbf{Примечание:} задание из книги Г. Вагнера «Основы исследования операций», т. 3, стр. 183 – 184
Крупная фирма, производящая моющие средства и пользующаяся широкой известностью в связи с успехами в исследованиях по созданию новых продуктов и их рекламированию, выпустила на рынок новый высококачественный стиральный порошок, названный LYE. Руководитель, возглавляющий производство этого продукта, совместно с отделом рекламы разрабатывает специальную рекламную кампанию по сбыту порошка, для которой принят девиз «Порошок LYE нужен всем!» Как и все продукты фирмы, новый продукт в течение первого полугодия будет иметь высокий уровень сбыта. Руководитель полагает, что с вероятностью 0,8 этот уровень сбыта сохранится и в последующем полугодии при условии проведения особой рекламной кампании и что эта вероятность составит всего 0,5, если такую кампанию не проводить. В случае, если уровень сбыта снизится до среднего, у руководителя имеются две возможности. Он может дать указание о проведении исследований с целью улучшения качества продукта. При этом условии с вероятностью 0,7 уровень сбыта к началу следующего полугодия повысится до первоначального высокого значения. С другой стороны, можно ничего не предпринимать в отношении улучшения качества продукта. Тогда с вероятностью 0,6 в начале последующего полугодия уровень сбыта останется средним, однако вследствие изменений потребительских вкусов он может вновь подняться до высокого значения лишь с вероятностью 0,4.

Если сбыт нового стирального порошка начинается на высоком уровне при обычной рекламе, то прибыли в течение полугодия равны 19 единицам в случае, когда этот уровень сохраняется, и равны 13, если уровень сбыта падает. При проведении специальной рекламной кампании соответствующие показатели равны 4,5 и 2 единицам. Если начальный уровень сбыта окажется средним и при этом проводятся исследования с целью улучшения качества продукции, то прибыли составят 11 единиц в случае, когда уровень сбыта поднимается до высокого, и 9 единиц в противном случае. При сохранении продукта в неизменном виде соответствующие прибыли равны 13 и 3 единицам. Предположим, что одна и та же проблема принятия решений относительно сбыта стирального порошка LYE повторяется через каждые полгода в течение бесконечного планового периода.


\section{Метод итераций по стратегиям}
Для начала выпишем все известные параметры задачи. 

Система может быть в двух состояних:
\begin{enumerate}
\item хороший сбыт($S_1$);
\item средний сбыт($S_2$).
\end{enumerate}  
Организация может предпринять следующие действия(далее стратегии):
\begin{enumerate}
\item всегда улучшать сбыт($X_1$)
\begin{itemize}
\item при $S_1$ - создание специальной рекламы($D_1$);
\item при $S_2$ - проведение исследований($D_2$). 
\end{itemize}
\item улучшать сбыт только при хорошем сбыте($X_2$)
\begin{itemize}
\item при $S_1$ - создание специальной рекламы($D_1$);
\item при $S_2$ - ничего не делать($D_3$). 
\end{itemize}
\item улучшать сбыт только при среднем сбыте($X_3$)
\begin{itemize}
\item при $S_1$ - ничего не делать($D_3$);
\item при $S_2$ - проведение исследований($D_2$). 
\end{itemize}
\item всегда ничего не делать($X_4$)
\begin{itemize}
\item при $S_1$ - ничего не делать($D_3$);
\item при $S_2$ - ничего не делать($D_3$). 
\end{itemize}
\end{enumerate}

На основе данной информации составим матрицы переходных вероятностей $P_1, P_2, P_3, P_4$ соответсвующие стратегиям $X_1, X_2, X_3, X_4$.

\begin{equation*}
P_1=\begin{pmatrix}
0.8 & 0.2\\
0.7 & 0.3
\end{pmatrix}\\
P_2=\begin{pmatrix}
0.8 & 0.2\\
0.4 & 0.6
\end{pmatrix}\\
P_3=\begin{pmatrix}
0.5 & 0.5\\
0.7 & 0.3
\end{pmatrix}\\
P_4=\begin{pmatrix}
0.5 & 0.5\\
0.4 & 0.6
\end{pmatrix}
\end{equation*}
%Строки означают попытки улучшить сбыт. Первая строка - специальная реклама, а вторая - улучшение качества продукта.
%
%Столбцы означают состояние сбытая. Первый столбец - хороший сбыт, второй - средний сбыт.
%
%То есть, если провести специальную рекламу, система с вероятностью в 0.8 останется в состоянии хорошого сбыта, и с верояностью в 0.2 перейдет в состояние среднего сбыта. По аналогии при улучшении качества продукта.
%
%Составим матрицу переходных вероятностей, если не предпринимать никаких действий.

%Если сравнивать $P_1$ и $P_2$, то у $P_1$ вероятность того что система будет в состоянии хорошего сбыта, выше чем у $P_2$.

Также составим матрицы доходов $R_1, R_2, R_3, R_4$.
\begin{equation*}
R_1=\begin{pmatrix}
4.5 & 2\\
11 & 9
\end{pmatrix}\\
R_2=\begin{pmatrix}
4.5 & 2\\
13 & 3
\end{pmatrix}\\
R_3=\begin{pmatrix}
19 & 13\\
11 & 9
\end{pmatrix}\\
R_4=\begin{pmatrix}
19 & 13\\
13 & 3
\end{pmatrix}
\end{equation*}
Множество допустимых стратегий $G=\{X_1, X_2, X_3, X_4\}$.
\subsection{Этап(1) оценивания параметров}
Выбираем стратегию $\tau$ - $X_4$.  Тогда, матрицы переходных вероятностей и доходов будут следующими:
\begin{equation*}
P=\begin{pmatrix}
0.5 & 0.5\\
0.4 & 0.6
\end{pmatrix}\\
R=\begin{pmatrix}
19 & 13\\
13 & 3
\end{pmatrix}
\end{equation*}
Учитывая, что  $F_{\tau}(2)=0$, получаем систему линейных алгебраических уравнений:

\begin{equation*}
\begin{cases}
E_{\tau}-(1-0.5)*F_{\tau}(1)=16\\
E_{\tau}-(1-0.4)*F_{\tau}(1)=7
\end{cases}
\end{equation*}
\begin{lstlisting}[language={matlab}, caption={Скрпит для решения системы уравнений},basicstyle=\ttfamily]
syms Et Ft
eqn1 = Et - (1-0.5)*Ft == 16;
eqn2 = Et - (1-0.4)*Ft == 7;
[A, B] = equationsToMatrix([eqn1, eqn2], [Et, Ft])
X = linsolve(A, B)
\end{lstlisting}
В результате выполнения скрипта matlab, было получено единственное решение:
\begin{equation*}
E_{\tau}=61; F_{\tau}(1)=90
\end{equation*}

\subsection{Этап(1) улучшения стратегии}
Для каждого состояния $S_j$, где $j$ от $1$ до $m$, найдем допустимое решение, на котором достигается:
\begin{equation*}
max(v_j(X_i)+\sum_{k=1}^mp_{jk}(X_i)F_{\tau}(k)
\end{equation*}


\tabulinesep = 1mm
\begin{longtabu} to \textwidth {|X[1, c , m ] |X[3, c , m ] | X[3,c , m ]|X[3,c , m ]|X[3,c , m ]| X[2,c , m ]|X[1,c , m ]|}\firsthline\hline

\multirow{2}{*}{$S_j$} & \multicolumn{4}{c|}{$\varphi_i=v_j(X_i)+p_{j1}(X_i)F_i(1)$} & \multirow{2}{*}{$max\varphi_i$} & \multirow{2}{*}{$X_{j}$}\\ \cline{2-5}
&i=1&i=2&i=3&i=4&&\\ \hline
1&4+0.8*(90)= 76&4+0.8*(90)= 76&16+0.5*(90)= 61&16+0.5*(90)= 61&76&$X_1$, $X_2$\\ \hline
2&10.4+0.7*(90)= 73.4&7+0.4*(90)= 40&10.4+0.7*(90)= 73.4&7+0.4*(90)= 40&73.4&$X_1$, $X_3$\\ \hline
\end{longtabu}


\subsection{Этап(2) оценивания параметров}
Выбираем стратегию $\tau$ - $X_3$.  Тогда, матрицы переходных вероятностей и доходов будут следующими:
\begin{equation*}
P=\begin{pmatrix}
0.5 & 0.5\\
0.7 & 0.3
\end{pmatrix}\\
R=\begin{pmatrix}
19 & 13\\
11 & 9
\end{pmatrix}
\end{equation*}
Учитывая, что  $F_{\tau}(2)=0$, получаем систему линейных алгебраических уравнений:

\begin{equation*}
\begin{cases}
E_{\tau}-(1-0.5)*F_{\tau}(1)=16\\
E_{\tau}-(1-0.7)*F_{\tau}(1)=10.4
\end{cases}
\end{equation*}
\begin{lstlisting}[language={matlab}, caption={Скрпит для решения системы уравнений},basicstyle=\ttfamily]
syms Et Ft
eqn1 = Et - (1-0.5)*Ft == 16;
eqn2 = Et - (1-0.7)*Ft == 10.4;
[A, B] = equationsToMatrix([eqn1, eqn2], [Et, Ft])
X = linsolve(A, B)
\end{lstlisting}
В результате выполнения скрипта matlab, было получено единственное решение:
\begin{equation*}
E_{\tau}=2; F_{\tau}(1)=-28
\end{equation*}

\subsection{Этап(2) улучшения стратегии}
Для каждого состояния $S_j$, где $j$ от $1$ до $m$, найдем допустимое решение, на котором достигается:
\begin{equation*}
max(v_j(X_i)+\sum_{k=1}^mp_{jk}(X_i)F_{\tau}(k)
\end{equation*}


\tabulinesep = 1mm
\begin{longtabu} to \textwidth {|X[1, c , m ] |X[3, c , m ] | X[3,c , m ]|X[3,c , m ]|X[3,c , m ]| X[2,c , m ]|X[1,c , m ]|}\firsthline\hline

\multirow{2}{*}{$S_j$} & \multicolumn{4}{c|}{$\varphi_i=v_j(X_i)+p_{j1}(X_i)F_i(1)$} & \multirow{2}{*}{$max\varphi_i$} & \multirow{2}{*}{$X_{j}$}\\ \cline{2-5}
&i=1&i=2&i=3&i=4&&\\ \hline
1&4+0.8*(-28)= -18.4&4+0.8*(-28)= -18.4&16+0.5*(-28)= 2&16+0.5*(-28)= 2&2&$X_3$, $X_4$\\ \hline
2&10.4+0.7*(-28)= -9.2&7+0.4*(-28)= -4.2&10.4+0.7*(-28)= -9.2&7+0.4*(-28)= -4.2&-4.2&$X_2$, $X_4$\\ \hline
\end{longtabu}

Так как $t\neq\tau$, то снова переходим к этапу оценивания параметров.


\subsection{Этап(3) оценивания параметров}
Выбираем стратегию $\tau$ - $X_2$.  Тогда, матрицы переходных вероятностей и доходов будут следующими:
\begin{equation*}
P=\begin{pmatrix}
0.8 & 0.2\\
0.4 & 0.6
\end{pmatrix}\\
R=\begin{pmatrix}
4.5 & 2\\
13 & 3
\end{pmatrix}
\end{equation*}
Учитывая, что  $F_{\tau}(2)=0$, получаем систему линейных алгебраических уравнений:

\begin{equation*}
\begin{cases}
E_{\tau}-(1-0.8)*F_{\tau}(1)=4\\
E_{\tau}-(1-0.4)*F_{\tau}(1)=7
\end{cases}
\end{equation*}
\begin{lstlisting}[language={matlab}, caption={Скрпит для решения системы уравнений}, basicstyle=\ttfamily]
syms Et Ft
eqn1 = Et - (1-0.8)*Ft == 4;
eqn2 = Et - (1-0.4)*Ft == 7;
[A, B] = equationsToMatrix([eqn1, eqn2], [Et, Ft])
X = linsolve(A, B)
\end{lstlisting}
В результате выполнения скрипта matlab, было получено единственное решение:
\begin{equation*}
E_{\tau}=5/2; F_{\tau}(1)=-15/2
\end{equation*}


\subsection{Этап(3) улучшения стратегии}
Для каждого состояния $S_j$, где $j$ от $1$ до $m$, найдем допустимое решение, на котором достигается:
\begin{equation*}
max(v_j(X_i)+\sum_{k=1}^mp_{jk}(X_i)F_{\tau}(k)
\end{equation*}


\tabulinesep = 1mm
\begin{longtabu} to \textwidth {|X[1, c , m ] |X[3, c , m ] | X[3,c , m ]|X[3,c , m ]|X[3,c , m ]| X[2,c , m ]|X[1,c , m ]|}\firsthline\hline

\multirow{2}{*}{$S_j$} & \multicolumn{4}{c|}{$\varphi_i=v_j(X_i)+p_{j1}(X_i)F_i(1)$} & \multirow{2}{*}{$max\varphi_i$} & \multirow{2}{*}{$X_{j}$}\\ \cline{2-5}
&i=1&i=2&i=3&i=4&&\\ \hline
1&4+0.8*(-15/2)= -2&4+0.8*(-15/2)= -2&16+0.5*(-15/2)= 12.25&16+0.5*(-15/2)= 12.25&12.25&$X_3$, $X_4$\\ \hline
2&10.4+0.7*(-15/2)= 5.15&7+0.4*(-15/2)= 4&10.4+0.7*(-15/2)= 5.15&7+0.4*(-15/2)= 4&5.15&$X_1$, $X_3$\\ \hline
\end{longtabu}

Так как $t\neq\tau$, то снова переходим к этапу оценивания параметров.


\subsection{Этап(4) оценивания параметров}
Выбираем стратегию $\tau$ - $X_1$.  Тогда, матрицы переходных вероятностей и доходов будут следующими:
\begin{equation*}
P=\begin{pmatrix}
0.8 & 0.2\\
0.7 & 0.3
\end{pmatrix}\\
R=\begin{pmatrix}
4.5 & 2\\
11 & 9
\end{pmatrix}
\end{equation*}
Учитывая, что  $F_{\tau}(2)=0$, получаем систему линейных алгебраических уравнений:

\begin{equation*}
\begin{cases}
E_{\tau}-(1-0.8)*F_{\tau}(1)=4\\
E_{\tau}-(1-0.7)*F_{\tau}(1)=10.4
\end{cases}
\end{equation*}
\begin{lstlisting}[language={matlab}, caption={Скрпит для решения системы уравнений}, basicstyle=\ttfamily]
syms Et Ft
eqn1 = Et - (1-0.8)*Ft == 4;
eqn2 = Et - (1-0.7)*Ft == 10.4;
[A, B] = equationsToMatrix([eqn1, eqn2], [Et, Ft])
X = linsolve(A, B)
\end{lstlisting}
В результате выполнения скрипта matlab, было получено единственное решение:
\begin{equation*}
E_{\tau}=-44/5; F_{\tau}(1)=-64
\end{equation*}

\subsection{Этап(4) улучшения стратегии}
Для каждого состояния $S_j$, где $j$ от $1$ до $m$, найдем допустимое решение, на котором достигается:
\begin{equation*}
max(v_j(X_i)+\sum_{k=1}^mp_{jk}(X_i)F_{\tau}(k)
\end{equation*}


\tabulinesep = 1mm
\begin{longtabu} to \textwidth {|X[1, c , m ] |X[3, c , m ] | X[3,c , m ]|X[3,c , m ]|X[3,c , m ]| X[2,c , m ]|X[1,c , m ]|}\firsthline\hline

\multirow{2}{*}{$S_j$} & \multicolumn{4}{c|}{$\varphi_i=v_j(X_i)+p_{j1}(X_i)F_i(1)$} & \multirow{2}{*}{$max\varphi_i$} & \multirow{2}{*}{$X_{j}$}\\ \cline{2-5}
&i=1&i=2&i=3&i=4&&\\ \hline
1&4+0.8*(-64)= -47.2&4+0.8*(-64)= -47.2&16+0.5*(-64)= -16&16+0.5*(-64)= -16&-16&$X_3$, $X_4$\\ \hline
2&10.4+0.7*(-64)= -34.4&7+0.4*(-64)= -18.6&10.4+0.7*(-64)= -34.4&7+0.4*(-64)= -18.6&-18.6&$X_2$, $X_4$\\ \hline
\end{longtabu}

Итак в этапе(2) и данном, были найдены оптимальные стратегии. То есть

\begin{equation*}
\tau=((X_3 \textit{ или } X_4), (X_2 \textit{ или } X_4))^T
\end{equation*}
В каждом из двух состояний, имеется два варианта дальнейших действий.

Если подвести итоги, то данное решение означает что:
\begin{itemize}
\item В состоянии хорошего сбыта($S_1$) - ничего не делать($D_3$);
\item В состоянии среднего сбыта($S_2$) - ничего не делать($D_3$).
\end{itemize}
И судя по данным итогам, ничего не делать($D_3$) является лучшей стратегией. Подобный исход можно объяснить тем, что при попытках увеличения сбыта, компания тратит на это деньги и соответственно доход снижается.


%\tabulinesep = 1mm
%\begin{longtabu} to \textwidth {|X[ c , m ] |X[c , m ] | X[ c , m ]|X[ c , m ]|X[ c , m ]|}\firsthline\hline
%1&\multicolumn{2}{|c|}{123}&4&5\\ \hline 
%\multicolumn{5}{|c|}{Оптимум $f_1$}\\ \hline
%
%
%\caption{Оптимумы критериев и значения функций}
%\end{longtabu}

%------------------------------------------------------------------------------

%\addcontentsline{toc}{section}{Список литературы}
%\bibliography{thesis}
%\bibliographystyle{ugost2008}

\end{document}