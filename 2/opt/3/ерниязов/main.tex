\documentclass[14pt,a4paper,report]{report}
\usepackage[a4paper, mag=1000, left=2.5cm, right=1cm, top=2cm, bottom=2cm, headsep=0.7cm, footskip=1cm]{geometry}
\usepackage[utf8]{inputenc}
\usepackage[english,russian]{babel}
\usepackage{indentfirst}
\usepackage[dvipsnames]{xcolor}
\usepackage[colorlinks]{hyperref}
\usepackage{listings} 
\usepackage{fancyhdr}
\usepackage{caption}
\usepackage{amsmath}
\usepackage{latexsym}
\usepackage{graphicx}
\usepackage{amsmath}
\usepackage{booktabs}
\usepackage{array}
\hypersetup{
	colorlinks = true,
	linkcolor  = black
}

\usepackage{titlesec}
\titleformat{\chapter}
{\Large\bfseries} % format
{}                % label
{0pt}             % sep
{\huge}           % before-code


\DeclareCaptionFont{white}{\color{white}} 

% Listing description
\usepackage{listings} 
\DeclareCaptionFormat{listing}{\colorbox{gray}{\parbox{\textwidth}{#1#2#3}}}
\captionsetup[lstlisting]{format=listing,labelfont=white,textfont=white}
\lstset{ 
	% Listing settings
	inputencoding = utf8,			
	extendedchars = \true, 
	keepspaces = true, 			  	 % Поддержка кириллицы и пробелов в комментариях
	language = Matlab,            	 	 % Язык программирования (для подсветки)
	basicstyle = \small\sffamily, 	 % Размер и начертание шрифта для подсветки кода
	numbers = left,               	 % Где поставить нумерацию строк (слева\справа)
	numberstyle = \tiny,          	 % Размер шрифта для номеров строк
	stepnumber = 1,               	 % Размер шага между двумя номерами строк
	numbersep = 5pt,              	 % Как далеко отстоят номера строк от подсвечиваемого кода
	backgroundcolor = \color{white}, % Цвет фона подсветки - используем \usepackage{color}
	showspaces = false,           	 % Показывать или нет пробелы специальными отступами
	showstringspaces = false,    	 % Показывать или нет пробелы в строках
	showtabs = false,           	 % Показывать или нет табуляцию в строках
	frame = single,              	 % Рисовать рамку вокруг кода
	tabsize = 2,                  	 % Размер табуляции по умолчанию равен 2 пробелам
	captionpos = t,             	 % Позиция заголовка вверху [t] или внизу [b] 
	breaklines = true,           	 % Автоматически переносить строки (да\нет)
	breakatwhitespace = false,   	 % Переносить строки только если есть пробел
	escapeinside = {\%*}{*)}      	 % Если нужно добавить комментарии в коде
}

\begin{document}

\def\contentsname{Содержание}

% Titlepage
\begin{titlepage}
	\begin{center}
		\textsc{Санкт-Петербургский Политехнический 
			Университет Петра Великого\\[5mm]
			Кафедра компьютерных систем и программных технологий}
		
		\vfill
		
		\textbf{Отчёт по лабораторной работе №3\\[3mm]
			Курс: «Методы оптимизации и принятия решений»\\[3mm]
			Тема: «Марковские модели принятия решений»\\[35mm]
			}
	\end{center}
	
	\hfill
	\begin{minipage}{.5\textwidth}
		Выполнил студент:\\[2mm] 
		Ерниязов Тимур Ертлеуевич\\
		Группа: 13541/2\\[5mm]
		
		Проверил:\\[2mm] 
		Сиднев Александр Георгиевич
	\end{minipage}
	\vfill
	\begin{center}
		Санкт-Петербург\\ \the\year\ г.
	\end{center}
\end{titlepage}

% Contents
\tableofcontents
\clearpage

\chapter{Лабораторная работа №3}

\section{Индивидуальное задание}

Студентка-общественница Н., которая учится в колледже, расположенном за городом, и живёт там в общежитии, субботние вечера проводит в городе, принимая участия в собраниях. Считая, что на собрания её должен сопровождать мужчина, она каждую неделю обращается с этой просьбой к одному из трёх своих приятелей-студентов В., Г. и Д. В случае отказа приятеля, к которому она обратилась, студентка отправляется в город одна. Известны условные вероятности согласия сопровождать Н. в зависимости от того, к кому она обращалась на предыдущей неделе. Условия поездки на Вероятность удовлетворения просьбы предыдущей неделе

\begin{table}[h!]
\begin{tabular}{|l|l|l|l|}
\hline
Условия поездки на предыдущей неделе &   В &   Г  & Д  \\ \hline
Без сопровождения & 0.5 & 0.5 & 0.75 \\ \hline
Вместе с В & 0.5 & 1/3 & 0.25 \\ \hline
 Вместе с Г &  0.25 & 2/3 & 1/3 \\ \hline
 Вместе с Д & 0.6 & 0.4 & 0.2 \\ \hline
\end{tabular}
\end{table}

До летних каникул осталось 4 недели. Необходимо определить стратегию, максимизирующую ожидаемое число случаев, когда студентка получает согласие любого из приятелей сопровождать её в город. 

\section{Ход работы}
Матрицы переходных вероятностей для каждого решения:


$ P_1=
\begin{pmatrix} 
0.5 & 0.5 & 0 & 0 \\
0.5 & 0.5 & 0 & 0\\
0.75 & 0.25 & 0 & 0\\
0.4  & 0.6 & 0 & 0
\end{pmatrix} $
$ P_2=
\begin{pmatrix} 
0.5 & 0 & 0.5 & 0 \\
2/3 & 0 & 1/3 & 0\\
1/3 & 0 & 2/3 & 0\\
0.6  & 0 & 0.4 & 0
\end{pmatrix} $
$ P_3=
\begin{pmatrix} 
0.25 & 0 & 0 & 0.75 \\
0.75 & 0 & 0 & 0.25\\
2/3 & 0 & 0 & 1/3\\
0.4  & 0 & 0 & 0.2
\end{pmatrix} $


Матрица доходов:
$ R =
\begin{pmatrix} 
0 & 1 &1 & 1 \\
0 & 1 & 1 & 1\\
0 & 1 & 1 & 1\\
0  & 1 & 1 & 1
\end{pmatrix} $

Вектора ожидаемого дохода для каждого решения:


$ V_1=
\begin{pmatrix} 
0.5  & 0.5 & 0.25 & 0.6
\end{pmatrix} $

$ V_2=
\begin{pmatrix} 
0.5  & 1/3 & 2/3 & 0.4
\end{pmatrix} $

$ V_3=
\begin{pmatrix} 
0.75  & 0.25 & 1/3 & 0.2
\end{pmatrix} $

Будем находить оптимальную стратегию по рекуррентному уравнению:

$$ f_i(j)= max(v_j(x_n_i)+\sum_{k=1}^m p_{jk} (i+1|x_n_i)f_{i+1}(j))    $$

, где   $i = 1,2,...N-1, j = 1,2,...,m$  .

\clearpage

\textbf{Этап 4}
\begin{table}[h!]
\begin{tabular}{|l|l|l|l|l|l|}
\hline
	 &	 k=1  &  k=2 &   k=3   & k \\ \hline
	без сопровождения   &  0.5000 &   0.5000  &  0.7500 & 3 \\
    c В &      0.5000 &   0.3333  &  0.2500 & 1 \\
    c Г &      0.2500 &   0.6667  &  0.3333 & 2 \\
    c Д&       0.6000 &   0.4000 &    0.2000 & 1 \\ \hline
\end{tabular}
\end{table}

\textbf{Этап 3}
\begin{table}[h!]
\begin{tabular}{|l|l|l|l|l|l|}
\hline
	 &	 k=1  &  k=2 &   k=3  &  k \\ \hline

       без сопровождения  &  1.1250  &  1.2083  &  1.3875 & 3 \\
    c В   &  1.1250  &  1.0556  &  0.9625 & 1 \\
       c Г   &  0.9375  &  1.3611  &  1.0333 & 2 \\
       c Д  &  1.2000  &  1.1167 &    0.9200 & 1 \\ \hline
		 \end{tabular}
\end{table}



\textbf{Этап 2}
\begin{table}[h!]
\begin{tabular}{|l|l|l|l|l|l|}
\hline
	 &	 k=1  &  k=2 &   k=3   & k \\ \hline

		 без сопровождения  &    1.7563 &    1.8743    & 1.9969 & 3\\
    c В     & 1.7563    & 1.7120    & 1.5906 & 1\\
     c Г & 1.5719    & 2.0366    & 1.6583 & 2\\
c Д   & 1.8300    & 1.7769    & 1.5500 & 1\\ \hline
		 \end{tabular}
\end{table}


\textbf{Этап 1}
	\begin{table}[h!]
\begin{tabular}{|l|l|l|l|l|l|}
\hline	 
	 &	 k=1  &  k=2 &   k=3   & k \\ \hline

     без сопровождения  &  2.3766  &  2.5167 &    2.6217 & 3\\
    c В   &  2.3766  &  2.3434 &    2.2052 & 1 \\
      c Г   &  2.1867  &  2.6900 &    2.2746 & 2\\ 
        c Д   &  2.4525  &  2.4128 &    2.1635 & 1\\ \hline
         \end{tabular}
\end{table}


Оптимальное решение показывает, что на протежении всех четерех недель следует придерживаться одной стратегии:
\begin{enumerate}
    \item если на предыдущей неделе студентка была без сопровождения - идти с Д
    \item если на предыдущей неделе студентка была с В - идти с В
    \item если на предыдущей неделе студентка была с Г - идти с Г
    \item если на предыдущей неделе студентка была с Д - идти с В. 
\end{enumerate}


Суммарная ожидаемая "выгода" за 4 месяца составит:
\begin{enumerate}
    \item без сопровождения - 2.6217 у.е.
    \item с В - 2.3766 у.е.
    \item с Г - 2.69 у.е.
    \item с Д -2.4525 у.е..
\end{enumerate}    

\begin{lstlisting}[caption = {Полный скрипт Matlab}]
n = 4;
v = []; p = []; f = []; x = [];

v(:, :) = [0 0.5 1/2 0.75;
           0 0.5 1/3 0.25;
           0 0.25 2/3 1/3;
           0 0.6 0.4 0.2;];


p(1, :, :) = [0.5 0.5 0 0;
              0.5 0.5 0 0;
              3/4 1/4 0 0;
              2/5 3/5 0 0];
              
p(2, :, :) = [0.5 0 0.5 0;
              2/3 0 1/3 0;
              1/3 0 2/3 0;
              0.6 0 0.4 0];
              
p(3, :, :) = [0.25 0 0 0.75;
              0.75 0 0 0.25;
              2/3 0 0 1/3;
              0.6 0 0 0.2];

f(:, n + 1) = zeros(4, 1);
for i = n:-1:1
    tmp = zeros(3);
    for j = 1:4
        for l = 1:3
            tmp(j, l) = v(j, l) + squeeze(p(l, j, :))'*f(:, i + 1)
        end
        [f(j, i), x(j, i)] = max(tmp(j, :));
    end
end
\end{lstlisting}


\section{Вывод}

Рассмотренная модель марковских процессов принятия решений позволяет решать задачи принятия решений в условиях риска при заданном одном критерии, либо нескольких, приведенных к одному. Система допущений, используемых в модели и приводящих реальную ситуацию к описанию с помощью
марковских процессов, ограничивает применение метода классом задач принятия решений, в которых можно принять допущение о дискретном времени, зависимости текущего состояния системы только от предшествующего и скачкообразном изменении состояния системы. Несмотря на эти ограничения, модель позволяет решать задачи принятия решений.

Основным методом решения марковских задач принятия решений в данной работе является рекурентный метод. Эта модель позволяет найти решения задачи за конечное число шагов, при этом не используя сложных и приближенных математических методов. Это облегчает поиск решения.

Решением задачи является вектор решений — стратегия — обеспечивающий оптимальное значение выбранного критерия.

\end{document}