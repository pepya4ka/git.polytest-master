\newpage
\section{Поиск оптимальной стратегии принятия решений с использованием марковских моделей.}

\subsection{Постановка задачи}
Пусть имеется машина (станок), которая обслуживается периодически один раз в час. В каждый момент она может находиться в одном из двух состояний: рабочем (состояние 1) и нерабочем (состояние 2).

Если машина на некотором шаге проработала непрерывно 1 час, то доход равен 3 рублям. При этом вероятность остаться на следующем шаге в состоянии 1 равна 0,7, а вероятность перейти в состояние 2 равна 0,3. Если машина отказала на некотором шаге, то её можно отремонтировать двумя способами. Первый является ускоренным, требует затрат в 2 рубля (доход равен -2 рубля) и обеспечивает переход в состояние 1 с вероятностью в 0,6. Второй, обычный способ требует затрат в 1 рубль и обеспечивает переход в состояние 1 с вероятностью 0,4.

Найти оптимальную стратегию для $N=\infty$ методом итераций по стратегиям, и перечислить все стационарные стратегии; построить марковскую модель принятия решений.

\subsection{Марковская модель принятия решений}

Матрицы переходных вероятностей ($P_1$ и $P_2$) и матрицы доходов ($r_1$ и $r_2$) имеют следующий вид:

\[ P_1 = \left( \begin{array}{cc}
0,7 & 0,3 \\
0,6 & 0,4
\end{array} \right)\qquad
%
P_2 = \left( \begin{array}{cc}
0,7 & 0,3 \\
0,4 & 0,6
\end{array} \right)
\]

\[ r_1 = \left( \begin{array}{rr}
3 & 0 \\
-2 & 0
\end{array} \right)\qquad
%
r_2 = \left( \begin{array}{rr}
3 & 0 \\
-1 & 0
\end{array} \right)
\]

Модель представлена на рисунке 3.

\begin{figure}[!h]
	\centering
	\begin{tikzpicture}[->,>=stealth',shorten >=1pt,auto,node distance=4cm,
	  thick,main node/.style={circle,fill=blue!20,draw,font=\sffamily\Large\bfseries}]

	  \node[main node] (1) {$S_1$};
	  \node[main node] (2) [below left of=1] {$D_1$};
	  \node[main node] (3) [below right of=2] {$S_2$};
	  \node[main node] (4) [below right of=1] {$D_2$};

	  \path[every node/.style={font=\sffamily\small}]
	    (1) edge node {0,3} (3)
	        edge [loop above] node {0,7} (1)
	    (2) edge [bend left] node [right] {0,6} (1)
	        edge node[right] {0,4} (3)
	    (3) edge [red, bend left] node [left] {2} (2)
	        edge [red, bend right] node[right] {1} (4)
	    (4) edge node [left] {0,6} (3)
	        edge [bend right] node[right] {0,4} (1);
	\end{tikzpicture}
	\caption{$S_1$ и $S_2$ состояния системы; $D_1$ и $D_2$ принимаемые решения; красные рёбра - траты, чёрные - вероятность перехода}
\end{figure}

После работы, машина можем:
\begin{itemize}
\item Остаться в исправном состоянии
$f^1 = \langle 1; 1 \rangle$
\item Перейти в неисправное состояние
$f^2 = \langle 1; 2 \rangle$
\end{itemize}

Таким образом, возможны следующие стационарные стратегии:
\begin{eqnarray*}
\pi_1^N = (f^1, f^1)\\
\pi_2^N = (f^1, f^2)\\
\pi_3^N = (f^2, f^1)\\
\pi_4^N = (f^2, f^2)
\end{eqnarray*}

\subsection{Метод итерации по стратегиям}

\textbf{Этап оценивания параметров}. Выбираем произвольную стратегию $\tau = (X_{j1}, X_{j2}, \dots, _{jm})^T$. Используя соответствующие стратегии $\tau$, матрицу переходных вероятностей $P(\tau) = (p_{ik}(\tau))$ и матрицу доходов $R(\tau) = (r_{jk}(\tau))$ и полагая $F_\tau(m) = 0$, решаем систему линейных алгебраических уравнений $E_\tau + F_\tau(j) - \sum\limits_{k=1}^m p_{jk}(\tau)F_\tau (k) = v_j (\tau)$, $j=\overline{1,m}$, относительно $E_\tau, F_\tau(1), \dots, F_\tau(m-1)$.

\textbf{Этап улучшения стратегии}. Для каждого состояния $S_j$, находим допустимое решение $X_{*j}$, на котором достигается $\text{max}_{(X_i \in G)}(v_j (X_i) + \sum\limits_{k=1}^m p_{jk}(X_i) F_\tau(k))$

Эти оптимальные решения образуют новую стратегию $t = (X_{*1}, X_{*2}, … X_{*m})^T$. Если $t = \tau$, то стратегия $\tau$ и является оптимальной. В противном случае нужно обозначить стратегию t через $\tau$ и вернуться к первому этапу.

Воспользовавшись матрицами $P_1$, $P_2$, $r_1$, $r_2$ и их независимостью от номера этапа, вычислим ожидаемые доходы, при различных вариантах допустимых решений:

\begin{eqnarray*}
v_1 (X_1 )=0,7*3 + 0,3*0=2,1		\\
v_2 (X_1 )=0,6*(-2) + 0,4*0=-1,2	\\
v_1 (X_2 )=0,7*3 + 0,3*0=2,1		\\
v_2 (X_2 )=0,4*(-1) + 0,6*0=-0,4
\end{eqnarray*}

В качестве произвольной стратегии $\tau$ используем стратегию номер два. В этом случае на этапе оценивания параметров, учитывая, что $F_\tau(2)=0$, получаем систему линейных алгебраических уравнений

$$
\left\{
	\begin{aligned}
	E_\tau + (1 - 0,7) F_\tau(1) &= & 2,1\\
	E_\tau - 0,6 &= & -1,2\\
	\end{aligned}
\right.
$$

которая имеет единственное решение: $E_\tau = 0,78$, $F_\tau(1) = 3,3$.

Результаты соответствующих вычислений приведены в табл. 5.

\begin{table}[htb]
	\begin{tabularx}{\textwidth}{|c|X|X|c|c|}
	\hline
	\multirow{2}{*}{$S_j$} & \multicolumn{2}{c|}{$\varphi (X_i )=v_j (X_k )+p_j1 (X_i)*F_i (1)$} & \multirow{2}{*}{max $\varphi_j$} & \multirow{2}{*}{$X_{*j}$} \\ 
	\hhline{~--~~}
	{} & i = 1 & i = 2 & {} & {} \\ 
	\hline 
	1 & 2,1+0,7*3,3=4,41 & 2,1+0,7*3,3=4,41 & 4,41 & $X_2$ \\ 
	\hline 
	2 & -1,2+0,6*3,3=0,78 & -0,4+0,4*3,3=0,92 & 0,78 & $X_2$ \\ 
	\hline
	\end{tabularx}
\caption{Решение с оценочными параметрами}
\end{table}

Новая стратегия $t = (X_2, X_2)^T$ отличается от стратегии $\tau$, поэтому нужно на этап оценивания параметров, полагая $\tau = (X_2, X_1)^T$.

Новой стратегии t соответствует следующая система линейных алгебраических уравнений:
$$
\left\{
	\begin{aligned}
	E_\tau + (1 - 0,7) F_\tau(1) &= & 2,1\\
	E_\tau - 0,4 &= & -0,4\\
	\end{aligned}
\right.
$$

которая имеет единственное решение: $E_\tau = 0,85$, $F_\tau(1) = 3,125$.

Результаты соответствующих вычислений приведены в табл. 6.

\begin{table}[htb]
	\begin{tabularx}{\textwidth}{|c|X|X|c|c|}
	\hline
	\multirow{2}{*}{$S_j$} & \multicolumn{2}{c|}{$\varphi (X_i )=v_j (X_k )+p_j1 (X_i)*F_i (1)$} & \multirow{2}{*}{max $\varphi_j$} & \multirow{2}{*}{$X_{*j}$} \\ 
	\hhline{~--~~}
	{} & i = 1 & i = 2 & {} & {} \\
	\hline
	1 & 2,1+0,7*3,125=4,2875 & 2,1+0,7*3,125=4,2875 & 4,2875 & $X_2$ \\
	\hline
	2 & -1,2+0,6*3,125=0,675 & -0,4+0,4*3,125=0,85 & 0,85 & $X_2$ \\
	\hline
	\end{tabularx}
\caption{Проверочное решение}
\end{table}

Новая стратегия совпала с предыдущей, таким образом оптимальная стратегия определена: оптимальное использовать более дешевый ремонт с меньшей гарантией успешного завершения.

\subsection{Метод линейного программирования}

Все параметры посчитаны, мы можем сформулировать задачу в виде задачи линейного программирования, для проверки ранее полученных результатов.

$$
\left\{
	\begin{aligned}
	2,1w_{11}+2,1w_{12}-1,2w_{21}-0,4w_{22} & ->& \text{max}\\
	0,3w_{11}+0,3w_{12}-0,6w_{21}-0,4w_{22} & = & 0 \\
	-0,3w_{11}-0.3w_{12}+0,6w_{21}+0,4w_{22} & = & 0 \\
	\sum\limits_{j=1}^2 \sum\limits_{i=1}^2 w_{ij}=1,\qquad w_{ij} \geq 0 &&\\
	\end{aligned}
\right.
$$

Код скрипта представлен в листинге 9.

\lstinputlisting[language=Matlab, caption={Код для вычисления задачи линейного программирования}]{../task3/linear.m}

\newpage

Результат выполнения в листинге 10.

\lstinputlisting[language={},caption={Результат работы скрипта линейного программирования}]{../task3/linear.out}

Таким образом, оптимальной стратегии снова стало использование дешевого ремонта, как и в предыдущем случае с итерации по стратегиям.