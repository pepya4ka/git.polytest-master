\newpage
\section{Поиск оптимальных параметров сети систем массового обслуживания.}

\subsection{Постановка задачи}
Минимизировать стоимость ССМО при ограничении на среднее число заявок в сети

\begin{gather*}
min\{F(u) = \sum\limits_{j=1}^n \sum\limits_{k=1}^{n_j} f_{jk} u_{jk} = \sum\limits_{j=1}^n \sum\limits_{k=1}^{n_j} (m_{jk} * \mu_{jk}) * u_{jk} \}
\end{gather*} 

$ L(u) = \sum\limits_{j=1}^n L_{jk} u_{jk}$,

$ u_{jk} =
  \begin{cases}
    0,\\
    1
 \end{cases}$,
 
Дано многоканальная сеть Джексона:

\begin{gather*}
\{ \lambda_0, \{jk\} - \text{набор альтернатив}, Q=\{q_{ij}\}_{i=\overline{0,n}, j=\overline{0,n}}, L(u) \}
\end{gather*} 

$L(u) = 4$, (предельное число заявок в сети)

$\lambda_0 = 6$,

$Q=\{q_{ij}\}_{i=\overline{0,n}, j=\overline{0,n}} = \begin{tabular}{|c|c|c|c|c|}
\hline 
0 & 0,2 & 0,7 & 0 & 0,1 \\ 
\hline 
0,1 & 0 & 0,1 & 0,2 & 0,6 \\ 
\hline 
0,5 & 0,1 & 0 & 0,2 & 0,2 \\ 
\hline 
0 & 0,2 & 0,8 & 0 & 0 \\ 
\hline 
0,3 & 0,2 & 0,2 & 0,3 & 0 \\ 
\hline 
\end{tabular} $.

Самостоятельно сформировать набор альтернатив (по 2 альтернативы на каждый узел, обеспечивающих установивший режим в узле).

Решить задачу 5 двумя способами:

\begin{itemize}
\item В соответствии с алгоритмом 5
\item Как задачу дискретного линейного программирования (например, с использованием Матлабовской команды Linprog).
\end{itemize}

\subsection{Алгоритм решения}

\begin{enumerate}
	\item Найти $\lambda_i, j = 1. \dots, m$ -- скорость прихода задач в узел $j$.\\
	$\lambda_{ij}=\lambda_i q_{ij}$	
	$\lambda_j = \lambda_{0j} + \sum\limits_{i=1}^n q_{ij}\lambda_i$, при $j=1,\dots,n$\\
	\item Возьмем произвольный набор альтернатив для каждого узла.\\
	$[(m_j^1,\mu_j^1),(m_j^1,\mu_j^1),\dots,(m_j^k,\mu_j^k)]$
	\item $p_j$ рассчитывается по следующей формуле\\
	$p_j = \frac{\lambda_j}{\mu_j m_j}$
	\item Вычислим $E(W_{qj})$:\\
	$E(W_{qj}) = \frac{(m_j p_j)^{m_j} \pi(0)}{\mu_j m_j (1-p_j)^2 m_j!}$, где
	$\pi(0) = \{\sum\limits_{t=0}^{m_j - 1} \frac{(m_j p_j)^t}{t!} + \frac{(m_j p_j) ^ m_j}{(1-p_j)m_j!}  \}^{-1}$
	\item Найдем $L_j$ -- число заявок в j-ом узле по формуле:\\
	$L_j = [E(W_{qj} + E(S_j)] \lambda_i$, где $E(S_j)= \frac{1}{\mu}$
\end{enumerate}

\subsection{Решение по алгоритму}

\textbf{Первый шаг алгоритма}

По первой формуле из первого шага алгоритма найдем вектор $\lambda_{0j}$.\\
$\lambda_{0j}= (1,2 \qquad 4,2 \qquad 0 \qquad 0,6)$

Используя вектор $\lambda_{0j}$, составим систему уравнений по второй формуле из шага 1.

$$
\left\{
	\begin{aligned}
	\lambda_0 &= & 6\\
	\lambda_1 &= & 1,2 + 0,2 * \lambda_0 +0,1 * \lambda_2 + 0,2 * \lambda_3 + 0,2 * \lambda_4\\
	\lambda_2 &= & 4,2 + 0,7* \lambda_0 + 0,1 * \lambda_1 +0,8 * \lambda_3 +0,2 * \lambda_4\\
	\lambda_3 &= & 0,2 * \lambda_1 +0,2 * \lambda_2 +0,3  * \lambda_4\\
	\lambda_4 &= & 0,6 + 0,1 * \lambda_0 + 0,6 * \lambda_1 + 0,2 * \lambda_2\\
	\end{aligned}
\right.
$$

В результате решения системы методом Гаусса был получен следующий результат:
\begin{itemize}
\item $\lambda_0 = 6 $;
\item $\lambda_1 = 7,44 $;
\item $\lambda_2 = 17,06 $;
\item $\lambda_3 = 7,63 $;
\item $\lambda_4 = 37,65 $.
\end{itemize}

Теперь вычислим остальные $\lambda_{ij}$, используя следующий скрипт. Здесь и в дальнейшем используется странное ограничение на предельное число заявок в сети -- 4 шт. вместе с нулевой.

\begin{Verbatim}[frame=single]
% Исходные данные
Q = [0 0.2 0.7 0 0.1;
    0.1 0 0.1 0.2 0.6;
    0.5 0.1 0 0.2 0.2;
    0 0.2 0.8 0 0;
    0.3 0.2 0.2 0.3 0];
% Значение лямбды каждого узла
lambdaj=[6;7.44;17.06;7.63;37.65];
N=4; % Число узлов вместе с нулевой.
% Вычисление лямбда ij
lambdaij=[];
 for i = 1:N
    for j = 1:N
         lambdaij(i,j) = lambdaj(i)*Q(i, j);
    end
end
lambdaij
\end{Verbatim}

Результаты вычисления представлены в таблице 7.

\begin{table}[htb]
	\begin{tabularx}{\textwidth}{|c|X|X|X|X|}
	\hline 
	i\textbackslash{}j & 1 & 2 & 3 & 4 \\ 
	\hline 
	1 &  0 & 1.2000 & 4.2000 & 0 \\ 
	\hline 
	2 & 0.7440 & 0 & 0.7440 & 1.4880 \\ 
	\hline
	3 & 8.5300 & 1.7060 & 0 & 3.4120 \\ 
	\hline 
	4 & 0 & 1.5260 & 6.1040 & 0 \\ 
	\hline 
	\end{tabularx}
\caption{Результат вычисления всех значений $\lambda_{ij}$}
\end{table}

\textbf{Второй шаг алгоритма}

Далее сформируем набор альтернатив (по 2 альтернативы на каждый узел, обеспечивающих установивший режим в узле):
\begin{itemize}
\item $m_j^1$=(4 8 3 4);
\item $\mu_j^1$=(2 3 7 9);
\item $m_j^2$=(5 8 3 6);
\item $\mu_j^2$=(9 13 10 4);
\end{itemize}

\textbf{Третий шаг алгоритма}

Для расчета вероятности $p_j$ воспользуемся следующим скриптом:
\begin{Verbatim}[frame=single]
K=2; % Задаем альтернативы
muj = [4 8 3 4;
       2 3 7 9];
 
mj = [5 8 3 6;
      9 13 10 4];

% Расчёт вероятностей
pj = [];
 for i = 1:N-1
    for k = 1:K
        pj(k,i) = lambdaj(i+1)/(muj(k,i)*mj(k,i));
    end
 end
pj
\end{Verbatim}

Результаты занесены в таблицу 8.

\begin{table}[htb]
	\begin{tabularx}{\textwidth}{|c|X|X|X|}
	\hline 
	k\textbackslash{}j & 1 & 2 & 3 \\ 
	\hline 
	1 & 0.3720 & 0.2666 & 0.8478 \\
	\hline 
	2 & 0.4133 & 0.4374 & 0.1090 \\
	\hline 
	\end{tabularx}
\caption{Результат расчёта вероятностей}
\end{table}

\textbf{Четвертый шаг алгоритма}

Расчет вектора $\pi_j (0)$.

\begin{Verbatim}[frame=single]
% Расчет начальной вероятности для каждого узла.
pij0 = [];
for i = 1:N-1
    for k=1:K
        sum = 0;
        for t = 0:mj(k,i)-1
            sum = sum + ((mj(k,i)*pj(k,i))^t)/factorial(t);
        end
        sum = sum + ((mj(k,i)*pj(k,i))^mj(k,i))/
                                    ((1-pj(k,i))*factorial(mj(k,i)));
        pij0(k,i)=sum^(-1);
    end
end
pij0
\end{Verbatim}

Результаты расчета начальной вероятности узлов занесены в таблицу 9.

\begin{table}[htb]
	\begin{tabularx}{\textwidth}{|c|X|X|X|}
	\hline 
	k\textbackslash{}j & 1 & 2 & 3 \\ 
	\hline 
	1 & 0.1549 & 0.1185 & 0.0403 \\ 
	\hline 
	2 & 0.0242 & 0.0034 & 0.3362 \\ 
	\hline 
	\end{tabularx}
\caption{Расчет начальной вероятности узлов}
\end{table}

\textbf{Пятый шаг алгоритма}

Рассчитаем $E(W_{q_j})$.

\begin{Verbatim}[frame=single]
% E(Wqj) для каждого узла.
EWqj = [];
for i = 1:N-1
    for k=1:K
        EWqj(k,i) = (((mj(k,i)*pj(k,i))^mj(k,i))*pij0(k,i))/
        	(muj(k,i)*mj(k,i)*((1-pj(k,i))^2)*factorial(mj(k,i)));
    end
end     
EWqj
\end{Verbatim}

Результаты занесены в таблицу 10.

\begin{table}[htb]
	\begin{tabularx}{\textwidth}{|c|X|X|X|}
	\hline 
	k\textbackslash{}j & 1 & 2 & 3 \\ 
	\hline
	1 & 0.0036 & 0.0000 & 0.5304 \\ 
	\hline 
	2 & 0.0015 & 0.0003 & 0.0000 \\ 
	\hline 
	\end{tabularx}
\caption{Расчет вектора $E(W_{q_j})$  для k=1 и k=2}
\end{table}

Теперь найдем $E(S_j)$ для каждого узла.
\begin{Verbatim}[frame=single]
% Найдем E(sj) для каждого узла.
Esj = [];
for i = 1:N-1
    for k=1:K
        Esj(k,i) = 1/muj(k,i);
    end
end  
Esj
\end{Verbatim}

Результаты занесены в таблицу 11.

\begin{table}[htb]
	\begin{tabularx}{\textwidth}{|c|X|X|X|}
	\hline 
	k\textbackslash{}j & 1 & 2 & 3 \\ 
	\hline 
	1 & 0.2500 & 0.1250 & 0.3333 \\ 
	\hline 
	2 & 0.5000 & 0.3333 & 0.1429 \\ 
	\hline 
	\end{tabularx}
\caption{Расчет $E(S_j)$ для каждого узла.}
\end{table}

Теперь не составит труда посчитать $L_j$ --  число заявок в j-ом узле, для каждой альтернативы.

\begin{Verbatim}[frame=single]
% Найдем Lj
Lj = [];
for i = 1:N-1
    for k=1:K
        Lj(k,i) = lambdaj(i+1)*(EWqj(k,i)+Esj(k,i));
    end
end
Lj
\end{Verbatim}

Результаты занесены в таблицу 12.

\begin{table}[htb]
	\begin{tabularx}{\textwidth}{|c|X|X|X|}
	\hline 
	k\textbackslash{}j & 1 & 2 & 3 \\ 
	\hline 
	1 & 1.8871 & 2.1331 & 6.5900 \\ 
	\hline 
	2 & 3.7309 & 5.6916 & 1.0900 \\ 
	\hline 
	\end{tabularx}
\caption{Расчет числа заявок в каждом узле.}
\end{table}

Таким образом, сеть не загружена и простаивает.

\subsection{Решение дискретным линейным методом программирования}

Для решения задачи воспользуемся функцией bintprog.

\begin{Verbatim}[frame=single]
%Зададим входные параметры
A = [1 0 0 1 0 0;
     0 1 0 0 1 0;
     0 0 1 0 0 1];
f = [4 4 5 5 2 31];
b = [1 1 1];
Aeq = A;
beq = [1;1;1];
[x, fval] = bintprog(f,A,b,Aeq,beq)
%параметры A,b,Aeq,beq будут как для прошлой задачи
%дополнительно введем ограничения на значение lb и ub
lb=zeros(1,(N-1)*2);
ub=ones(1,(N-1)*2);
%решаем задачу
[x,fval] = linprog(f,A,b,Aeq,beq,lb,ub)
\end{Verbatim}

Результаты представлены в таблице 13.

\begin{table}[htb]
	\begin{tabularx}{\textwidth}{|c|X|X|X|}
	\hline 
	k & $u_{1k}$ & $u_{2k}$ & $u_{3k}$ \\ 
	\hline 
	1 & 1 & 1 & 1 \\ 
	\hline 
	2 & 0 & 0 & 0 \\ 
	\hline 
	\end{tabularx}
\caption{Результат использования linprog.}
\end{table}

Получаем следующий выбор альтернатив: ($u_{11}$, $u_{21}$, $u_{31}$).
