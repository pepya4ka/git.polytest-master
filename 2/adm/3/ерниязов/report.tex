\documentclass[14pt,a4paper,report]{report}
\usepackage[a4paper, mag=1000, left=2.5cm, right=1cm, top=2cm, bottom=2cm, headsep=0.7cm, footskip=1cm]{geometry}
\usepackage[utf8]{inputenc}
\usepackage[english,russian]{babel}
\usepackage{indentfirst}
\usepackage[dvipsnames]{xcolor}
\usepackage[colorlinks]{hyperref}
\usepackage{listings} 
\usepackage{fancyhdr}
\usepackage{caption}
\usepackage{amsmath}
\usepackage{latexsym}
\usepackage{graphicx}
\usepackage{amsmath}
\usepackage{booktabs}
\usepackage{array}
\hypersetup{
	colorlinks = true,
	linkcolor  = black
}

\usepackage{titlesec}
\titleformat{\chapter}
{\Large\bfseries} % format
{}                % label
{0pt}             % sep
{\huge}           % before-code


\DeclareCaptionFont{white}{\color{white}} 

% Listing description
\usepackage{listings} 
\DeclareCaptionFormat{listing}{\colorbox{gray}{\parbox{\textwidth}{#1#2#3}}}
\captionsetup[lstlisting]{format=listing,labelfont=white,textfont=white}
\lstset{ 
	% Listing settings
	inputencoding = utf8,			
	extendedchars = \true, 
	keepspaces = true, 			  	 % Поддержка кириллицы и пробелов в комментариях
	language = C++,            	 	 % Язык программирования (для подсветки)
	basicstyle = \small\sffamily, 	 % Размер и начертание шрифта для подсветки кода
	numbers = left,               	 % Где поставить нумерацию строк (слева\справа)
	numberstyle = \tiny,          	 % Размер шрифта для номеров строк
	stepnumber = 1,               	 % Размер шага между двумя номерами строк
	numbersep = 5pt,              	 % Как далеко отстоят номера строк от подсвечиваемого кода
	backgroundcolor = \color{white}, % Цвет фона подсветки - используем \usepackage{color}
	showspaces = false,           	 % Показывать или нет пробелы специальными отступами
	showstringspaces = false,    	 % Показывать или нет пробелы в строках
	showtabs = false,           	 % Показывать или нет табуляцию в строках
	frame = single,              	 % Рисовать рамку вокруг кода
	tabsize = 2,                  	 % Размер табуляции по умолчанию равен 2 пробелам
	captionpos = t,             	 % Позиция заголовка вверху [t] или внизу [b] 
	breaklines = true,           	 % Автоматически переносить строки (да\нет)
	breakatwhitespace = false,   	 % Переносить строки только если есть пробел
	escapeinside = {\%*}{*)}      	 % Если нужно добавить комментарии в коде
}

\begin{document}

\def\contentsname{Содержание}

% Titlepage
\begin{titlepage}
	\begin{center}
		\textsc{Санкт-Петербургский Политехнический 
			Университет Петра Великого\\[5mm]
			Кафедра компьютерных систем и программных технологий}
		
		\vfill
		
		\textbf{Отчёт по лабораторной работе №3\\[3mm]
			Курс: «Администрирование компьютерных сетей»\\[3mm]
			Тема: «Администрирование сетевых сервисов»\\[35mm]
			}
	\end{center}
	
	\hfill
	\begin{minipage}{.5\textwidth}
		Выполнил студент:\\[2mm] 
		Ерниязов Тимур Ертлеуевич\\
		Группа: 13541/2\\[5mm]
		
		Проверил:\\[2mm] 
		Малышев Игорь Алексеевич
	\end{minipage}
	\vfill
	\begin{center}
		Санкт-Петербург\\ \the\year\ г.
	\end{center}
\end{titlepage}

% Contents
\tableofcontents
\clearpage

\chapter{Лабораторная работа №3}
\section{Цели работы}
\begin{enumerate}
\item Изучение состава и функциональных возможностей сетевых сервисов операционных систем.
\item Разработка и настройка сервисов локальной сети.
\item Разработка и настройка сервисов демилитаризованной зоны.
\item Разработка и настройка сервисов пограничной зоны.
\end{enumerate}

\section{Сведения о системе}
Работа производилась на реальной системе, со следующими характеристиками:

\begin{center}
\begin{tabular}{ c c c }
Элемент&Значение\\  
	
Имя ОС&Майкрософт Windows 10 Pro (Registered Trademark)\\ 
Версия&10.0.16299 Сборка 16299\\ 
RAM &16,00 ГБ\\ 
Процессор&Intel(R) Core(TM) i5-7300HQ CPU @ 2.50GHz, 2496 МГц, ядер: 4, логических процессоров: 4\\ 
\end{tabular}
\end{center}


Для выполнения работы использовалась \textbf{VMware Workstation 12 pro (12.5.7 build-5813279)}



\section{Структура сети}
В качестве сети для экспериментов, использовалась ККС из прошлой работы, ОС семейства Windows не использовались. Были добавлены две виртуальных машины с ОС преведенными далее.
\begin{center}
\begin{tabular}{ c c c }
Название&Версия&Объем RAM\\  
Ubuntu&16.04 amd64&1.5 GB\\ 
Linux mint&18.3 Cinnamon 64 bit&1.5 GB\\ 
\end{tabular}
\end{center}

Также к сетям \textbf{VMnet2} и \textbf{VMnet3} были подключены бездисковые клиенты.

\begin{center}
\begin{tabular}{ c c c c}
Название сети&Адрес сети&Подключенные узлы&DHCP\\  	
VMnet1&192.168.40.0&Linux mint, NetBSD, FreeBSD&-\\ 
VMnet2&192.168.80.0&FreeBSD, NoDisk1&-\\ 
VMnet3&192.168.120.0&FreeBSD, Ubuntu, NoDisk2&-\\ 
VMnet4&192.168.32.0&NetBSD&+\\ 
\end{tabular}
\end{center}

\section{DHCP и удаленная загрузка}
В данной работе используются два DHCP сервера, на FreeBSD и на Ubuntu, для двух бездисковых клиентов из разных сетей. Также на Ubuntu будет установлен TFTP сервер для удаленной загрузки ОС.
\subsection{Настройка DHCP на FreeBSD}
\begin{enumerate}
\item Переходим в каталог \textbf{/usr/ports/net/isc-dhcp3-server}
\item Устанавливаем следующей командой:
\begin{lstlisting}[language={}]
make install clean
\end{lstlisting}
\item Вносим в файл \textbf{/etc/rc.conf} следующие строки:
\begin{lstlisting}[language={}]
# Включаем DHCP
dhcpd_enable="YES"
# Отлючаем вывод избыточной информации
dhcpd_flags="-q" 
# Указываем интерфейс для запуска
dhcpd_ifaces="em1" 
\end{lstlisting}
\item Создаем файл \textbf{/usr/local/etc/dhcpd.conf} и вносим в него следующие изменения:
\begin{lstlisting}[language={}]
option domain-name "example.org"; # доменное имя
option domain-name-servers 192.168.32.2; #DNS сервер
default-lease-time 600; 
max-lease-time 7200; 

subnet 192.168.80.0 netmask 255.255.255.0 {
    range 192.168.80.127 192.168.80.224;
    option routers 192.168.80.2;
    option root-path "192.168.120.3:/usr/tftpboot/";
    next-server 192.168.120.3;
    filename "gpxelinux.0";
}
\end{lstlisting}
\end{enumerate}
\subsection{Настройка DHCP на Ubuntu}
Для начала необходимо корректно сконфигурировать сетеов адаптер для выхода в сеть "Интернет". Для этого были заданы следующие параметры:
\begin{itemize}
\item Адрес - 192.168.120.3;
\item Маска - 255.255.255.0;
\item Шлюз - 192.168.120.2;
\item DNS сервер - 192.168.32.2.
\end{itemize}
После настройки адапетра переходим к основной настройке TFTP сервера.
\begin{enumerate}
\item Выполняем команду:
\begin{lstlisting}[language={}]
sudo apt-get install isc-dhcp-server
\end{lstlisting}
\item В файл \textbf{/etc/dhcp/dhcpd.conf} вносим следующие изменения:
\begin{lstlisting}[language={}]
option domain-name "example.org";
option domain-name-servers 192.168.32.2;

default-lease-time 600;
max-lease-time 7200;

subnet 192.168.120.0 netmask 255.255.255.0 {
    range 192.168.120.100 192.168.120.200;
    option routers 192.168.120.2;
    next-server 192.168.120.3;  # TFTP server address
    filename "gpxelinux.0";   # PXE boot loader filename
}
\end{lstlisting}
\item Перезапускаем DHCP-сервер
\begin{lstlisting}[language={}]
sudo /etc/init.d/isc-dhcp-server restart
\end{lstlisting}
\end{enumerate}

\subsection{Настройка TFTP на Ubuntu}
\begin{enumerate}
\item Выполняем установку пакетов командой:
\begin{lstlisting}[language={}]
sudo apt-get install tftp tftpd-hpa
\end{lstlisting}
\item Создаем необходимые для работы директории:
\begin{lstlisting}[language={}]
mkdir -p /usr/tftpboot/images
mkdir /usr/tftpboot/pxelinux.cfg
\end{lstlisting}
\item В файл \textbf{/etc/rc.conf} вносим изменения:
\begin{lstlisting}[language={}]
tftpd_enable="YES"
tftpd_flags="-p -s /usr/tftpboot -B 1024 --ipv4"
\end{lstlisting}
\item Запускаем TFTP-сервер
\begin{lstlisting}[language={}]
service tftpd start
\end{lstlisting}
\item Скачиваем Syslinux версии 6.03 с сайта:
\begin{lstlisting}[language={}]
http://www.syslinux.org/wiki/index.php?title=The_Syslinux_Project
\end{lstlisting}
\item Извлекаем по пути \textbf{/usr/tftpboot} следующие файлы:

\begin{itemize}
\item chain.c32
\item gpxelinux.0
\item ldlinux.c32
\item libutil.c32
\item memdisk
\item menu.c32
\item reboot.c32
\item vesamenu.c32
\end{itemize}

\item Скачиваем образ ОС, поддерживающей liveCD, например ubuntu;
\item Разархивируем образ ОС, по пути \textbf{/usr/tftpboot/images/ubuntu/};
\item По пути \textbf{/usr/tftpboot/pxelinux.cfg} создаем файл \textbf{default} со следующим содержимым:
\begin{lstlisting}[language={}]
ui menu.c32
menu title Netboot OS

LABEL ubuntu
        kernel images/ubuntu/casper/vmlinuz.efi
        append root=/dev/nfs boot=casper netboot=nfs nfsroot=192.168.120.3:/usr/tftpboot/images/ubuntu initrd=images/ubuntu/casper/initrd.lz 
\end{lstlisting}
\item Перезгружаем ОС.
\end{enumerate}

\subsection{Проверка}
Запускаем бездисковые клиенты. При запуске вспывет меню, с выбором дальнейшей загрузки, необходимо выбрать пункт с \textbf{Ubuntu}. Далее, при успешной загрузке попадаем на рабочий стол ОС, что говорит об успешной удаленной загрузке. Дополнительно можно посмотротеть с помощью команды \textbf{ifconfig} какой адрес был выдан DHCP сервером. 

При корректных настройках, адрес будет находиться в диапазоне адресов, указанных на DHCP сервере, а также клиент будет иметь возможность выхода в сеть "Интернет".

\section{DNS сервисы}
В данном разделе будут приведены примеры различных DNS серверов, в частности:
\begin{enumerate}
\item \textbf{Кэширующий сервер} - ищет все ответы на запросы от пользователей и запоминает их на случай повторного запроса
\item \textbf{Первичный мастер} - читает данные зоны из локального файла и является ответственным за эту зону
\item \textbf{Вторичный мастер} - получает данные по зоне с другого сервера имен,
отвечающего за эту зону
\end{enumerate}

\subsection{Кэширующий сервер}
\subsubsection{Настройка}
Установим кэширующий DNS на Linux Mint(192.168.40.32)
\begin{enumerate}
\item Устанавливаем пакет \textbf{bind} командой
\begin{lstlisting}[language={}]
sudo apt-get install bind9
\end{lstlisting}
\item Внесем изменения в файл \textbf{/etc/bind/named.conf.options}:
\begin{lstlisting}[language={}]
options {
  directory "/var/cache/bind";
  forwarders {
    8.8.8.8;
  };
  dnssec-validation auto;
  auth-nxdomain no;
  listen-on-v6 {any;};
};
\end{lstlisting}
\item Перезапускаем DNS сервер
\begin{lstlisting}[language={}]
sudo /etc/init.d/bind9 restart
\end{lstlisting}
\end{enumerate}
\subsubsection{Проверка}
Настроим сетевой адаптер хост \textbf{192.168.120.3} на использование данного DNS сервера. В файле \textbf{/etc/NetworkManager/NetworkManager.conf} комментируем строку \textbf{dns=dnsmasq}, иначе в nslookup будет другой адрес DNS. Перезагружаем систему. Вводим команду \textbf{nslookup ya.ru}, после чего в консоль будет выведено:
\begin{lstlisting}[language={}]
Server:        192.168.40.32
Address:       192.168.40.32#53

Non-authoritative answer:
Name:    ya.ru
Address: 87.250.250.242
\end{lstlisting}
Как видно, DNS успешно был определен.

Дополнитьно проверим кэширующую способность DNS сервера, для этого введем команду \textbf{dig google.com} два раза:
\begin{lstlisting}[language={}]
dig google.com
...
;; Query time: 31 msec
;; SERVER: 192.168.40.32#53(192.168.40.32)
;; WHEN: Fri Mar 23 11:55:03 PDT 2018
;; MSG SIZE  rcvd: 346


dig google.com
...
;; Query time: 0 msec
;; SERVER: 192.168.40.32#53(192.168.40.32)
;; WHEN: Fri Mar 23 11:55:03 PDT 2018
;; MSG SIZE  rcvd: 346
\end{lstlisting}
Как видно, время запроса снизилось с 32 до 0 милисекунд, что означает о наличии кэширующей возможности у DNS сервера.

\subsection{Первичный и вторичный DNS серверы}
Первичный сервер - Linux mint \textbf{192.168.40.32}, вторичный - Ubuntu \textbf{192.168.120.3}.
\subsubsection{Настройка первичного DNS сервера}
В файл \textbf{/etc/bind/named.conf.local} вносим изменения:
\begin{lstlisting}[language={}]
zone "example.com" {
  type master;
  file "/etc/bind/db.example.com";
  allow-transfer {192.168.40.0/24;192.168.80.0/24;192.168.120.0/24;};
};
zone "40.168.192.in-addr.arpa" {
  type master;
  file "/etc/bind/db.192";
  allow-transfer {192.168.40.0/24;192.168.80.0/24;192.168.120.0/24;};
}; 
\end{lstlisting}
Были созданы зоны \textbf{example.com} и обратная ей зона(для получения имени по IP)

Содержимое файла \textbf{db.example.com} выглядит следующим образом:
\begin{lstlisting}[language={}]
;
; BIND data file for example.com
;
$TTL    604800
@   IN   SOA   example.com. root.example.com. (
             1          ; Serial
             604800     ; Refresh
              86400     ; Retry
            2419200     ; Expire
             604800 )   ; Negative Cache TTL
    IN A   192.168.40.32
;
@   IN  NS  ns.example.com
@   IN  A  192.168.40.32
@   IN  AAAA    ::1
ns  IN  A  192.168.40.32
\end{lstlisting}

Содержимое файла \textbf{db.192} выглядит следующим образом:
\begin{lstlisting}[language={}]
;
; BIND reverse data file for local 192.168.40.XXX net
;
$TTL    604800
@       IN      SOA     ns.example.com. root.example.com. (
                         1              ; Serial
                         604800         ; Refresh
                          86400         ; Retry
                        2419200         ; Expire
                         604800 )       ; Negative Cache TTL
;
@       IN      NS      ns.
32      IN      PTR     ns.example.com.
\end{lstlisting}
В данном случае адресу 192.168.40.32 соответствует имя example.com. 

\subsubsection{Настройка вторичного DNS сервера}
\begin{enumerate}
\item Устанавливаем пакет bind9;
\item В файл \textbf{/etc/bind/named.conf.local} вносим изменения:
\begin{lstlisting}[language={}]
zone "example.com" {
  type slave;
  file "db.example.com";
  masters { 192.168.40.32; };
};        
zone "40.168.192.in-addr.arpa" {
  type slave;
  file "db.192";
  masters { 192.168.40.32; };
};
\end{lstlisting}
\end{enumerate}
Теперь данный сервер является вторичным по отношению к 192.168.40.32.

\subsubsection{Проверка}
На первичном сервере вводим команду \textbf{cat /etc/log/syslog}
\begin{lstlisting}[language={}]
...
... zone example.com/IN: loaded serial 1
... zone 40.168.192.in-addr.arpa/IN: loaded serial 1
... zone 255.in-addr.arpa/IN: loaded serial 1
... zone localhost/IN: loaded serial 2
... all zones loaded
... running
... zone 40.168.192.in-addr.arpa/IN: sending notifies (serial 1)
...
\end{lstlisting}
Как видно из части лога, зоны были успешно загружены и переданы на вторичный сервер.


\section*{Вывод}
В данной работе был получен опыт по настройке DHCP и TFTP серверов, что позволяет загружать по сети ОС или утилиту. 

Также были применены навыки работы с различными по функциональности DNS серверами. Так был получен кэширующий DNS сервер, который существенно ускоряет обращение к сетевым ресурсам. Также были сконфигурированы первичный и вторичный DNS сервера, ответственные за домен example.com и обратную зону.
%------------------------------------------------------------------------------

%\addcontentsline{toc}{section}{Список литературы}
%\bibliography{thesis}
%\bibliographystyle{ugost2008}

\end{document}