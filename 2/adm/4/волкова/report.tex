\documentclass[14pt,a4paper,report]{report}
\usepackage[a4paper, mag=1000, left=2.5cm, right=1cm, top=2cm, bottom=2cm, headsep=0.7cm, footskip=1cm]{geometry}
\usepackage[utf8]{inputenc}
\usepackage[english,russian]{babel}
\usepackage{indentfirst}
\usepackage[dvipsnames]{xcolor}
\usepackage[colorlinks]{hyperref}
\usepackage{listings} 
\usepackage{fancyhdr}
\usepackage{caption}
\usepackage{amsmath}
\usepackage{latexsym}
\usepackage{graphicx}
\usepackage{amsmath}
\usepackage{booktabs}
\usepackage{array}
\hypersetup{
	colorlinks = true,
	linkcolor  = black
}

\usepackage{titlesec}
\titleformat{\chapter}
{\Large\bfseries} % format
{}                % label
{0pt}             % sep
{\huge}           % before-code


\DeclareCaptionFont{white}{\color{white}} 

% Listing description
\usepackage{listings} 
\DeclareCaptionFormat{listing}{\colorbox{gray}{\parbox{\textwidth}{#1#2#3}}}
\captionsetup[lstlisting]{format=listing,labelfont=white,textfont=white}
\lstset{ 
	% Listing settings
	inputencoding = utf8,			
	extendedchars = \true, 
	keepspaces = true, 			  	 % Поддержка кириллицы и пробелов в комментариях
	language = C++,            	 	 % Язык программирования (для подсветки)
	basicstyle = \small\sffamily, 	 % Размер и начертание шрифта для подсветки кода
	numbers = left,               	 % Где поставить нумерацию строк (слева\справа)
	numberstyle = \tiny,          	 % Размер шрифта для номеров строк
	stepnumber = 1,               	 % Размер шага между двумя номерами строк
	numbersep = 5pt,              	 % Как далеко отстоят номера строк от подсвечиваемого кода
	backgroundcolor = \color{white}, % Цвет фона подсветки - используем \usepackage{color}
	showspaces = false,           	 % Показывать или нет пробелы специальными отступами
	showstringspaces = false,    	 % Показывать или нет пробелы в строках
	showtabs = false,           	 % Показывать или нет табуляцию в строках
	frame = single,              	 % Рисовать рамку вокруг кода
	tabsize = 2,                  	 % Размер табуляции по умолчанию равен 2 пробелам
	captionpos = t,             	 % Позиция заголовка вверху [t] или внизу [b] 
	breaklines = true,           	 % Автоматически переносить строки (да\нет)
	breakatwhitespace = false,   	 % Переносить строки только если есть пробел
	escapeinside = {\%*}{*)}      	 % Если нужно добавить комментарии в коде
}

\begin{document}

\def\contentsname{Содержание}

% Titlepage
\begin{titlepage}
	\begin{center}
		\textsc{Санкт-Петербургский Политехнический 
			Университет Петра Великого\\[5mm]
			Кафедра компьютерных систем и программных технологий}
		
		\vfill
		
		\textbf{Отчёт по лабораторной работе №4\\[3mm]
			Курс: «Администрирование компьютерных сетей»\\[3mm]
			Тема: «Устранение уязвимостей»\\[35mm]
			}
	\end{center}
	
	\hfill
	\begin{minipage}{.5\textwidth}
		Выполнил студент:\\[2mm] 
		Волкова Мария Дмитриевна\\
		Группа: 13541/2\\[5mm]
		
		Проверил:\\[2mm] 
		Малышев Игорь Алексеевич
	\end{minipage}
	\vfill
	\begin{center}
		Санкт-Петербург\\ \the\year\ г.
	\end{center}
\end{titlepage}

% Contents
\tableofcontents
\clearpage

\chapter{Лабораторная работа №4}
\section{Цели работы}
\begin{enumerate}
\item Устранение найденных уязвимостей хостов в сети.
\end{enumerate}

\section{Выполнение работы}
Ранее были выявлены следующие уязвимости:
\begin{itemize}
\item Windows XP (192.168.80.128)
\begin{itemize}
\item сервисом NTP открыт порт 123 по UDP;
\item сервисом RPC Windows открыт порт 135 по TCP;
\item сервисом NBNS открыт порт 137 по UDP;
\item сервисом NetBIOS открыт порт 139 по TCP.
\end{itemize}
\item Windows 98 (192.168.120.15)
\begin{itemize}
\item сервисом NetBIOS-SSN открыт порт 137 по UDP;
\item сервисом NetBIOS открыт порт 139 по TCP.
\end{itemize}
\end{itemize}

Устранить данные уязвимости можно следующими способами:
\begin{enumerate}
\item установка патчей безопасности;
\item редактирование реестра и системных настроек;
\item настройка межсетевого экрана.
\end{enumerate}
В данном случае будет настроен межсетевой экран, который будет настроен на хосте с FreeBSD.\\\\
На узле с FreeBSD, внесем следующие строчки в файл \textbf{/etc/rc.conf}:
\begin{lstlisting}[language={}]
firewall_enable="YES"
firewall_type="/usr/fw_rules/ruleFile"
\end{lstlisting}
Первой строчкой включается межсетевой экран, а второй файл с фильтрами.\\\\
В созданный файл \textbf{/usr/fw\_rules/ruleFile} внесем следующие строчки:
\begin{lstlisting}[language={}]
# Windows XP
add deny udp from any to 192.168.80.128 123
add deny tcp from any to 192.168.80.128 135
add deny udp from any to 192.168.80.128 137
add deny tcp from any to 192.168.80.128 139

# Windows 98
add deny udp from any to 192.168.120.15 137
add deny tcp from any to 192.168.120.15 139

# Allow all other packets
add allow in
add allow out
\end{lstlisting}
Данными правилами были заблокированы пакеты, которые могут использовать уязвимости. Применения првил идет сверху вниз, поэтому остальные пакеты проигнорируют правила для узлов с уязвимостями и к ним будут применены правила типа \textbf{add allow in} или \textbf{add allow out} тоесть пропуск всех исходящих и входящих пакетов.

После этого, для применения правил, необходимо перезагрузить систему.\\\\
В качестве проверки можно использовать утилиту \textbf{telnet}. 

Например команда \textbf{telnet 192.168.80.128 135} успешно установит соединение с уязвимым портом до включения межсетевого экрана. После включения, как и ожидается, соединение установить не получится, что говорит о невозможности использовать данный порт как уязвимость.

\section*{Вывод}
В данной работе были рассмотрены возможности по устранению уязвимостей ОС.

Наилучшим решением будет являться установка последних патчей безопасности, но это не всегда возможно. 

Более оперативным и гибким решением является настройка межсетевого экрана, что и было сделано в данной работе.
%------------------------------------------------------------------------------

%\addcontentsline{toc}{section}{Список литературы}
%\bibliography{thesis}
%\bibliographystyle{ugost2008}

\end{document}