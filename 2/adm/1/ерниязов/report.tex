\documentclass[14pt,a4paper,report]{report}
\usepackage[a4paper, mag=1000, left=2.5cm, right=1cm, top=2cm, bottom=2cm, headsep=0.7cm, footskip=1cm]{geometry}
\usepackage[utf8]{inputenc}
\usepackage[english,russian]{babel}
\usepackage{indentfirst}
\usepackage[dvipsnames]{xcolor}
\usepackage[colorlinks]{hyperref}
\usepackage{listings} 
\usepackage{fancyhdr}
\usepackage{caption}
\usepackage{amsmath}
\usepackage{latexsym}
\usepackage{graphicx}
\usepackage{amsmath}
\usepackage{booktabs}
\usepackage{array}
\hypersetup{
	colorlinks = true,
	linkcolor  = black
}

\usepackage{titlesec}
\titleformat{\chapter}
{\Large\bfseries} % format
{}                % label
{0pt}             % sep
{\huge}           % before-code


\DeclareCaptionFont{white}{\color{white}} 

% Listing description
\usepackage{listings} 
\DeclareCaptionFormat{listing}{\colorbox{gray}{\parbox{\textwidth}{#1#2#3}}}
\captionsetup[lstlisting]{format=listing,labelfont=white,textfont=white}
\lstset{ 
	% Listing settings
	inputencoding = utf8,			
	extendedchars = \true, 
	keepspaces = true, 			  	 % Поддержка кириллицы и пробелов в комментариях
	language = C++,            	 	 % Язык программирования (для подсветки)
	basicstyle = \small\sffamily, 	 % Размер и начертание шрифта для подсветки кода
	numbers = left,               	 % Где поставить нумерацию строк (слева\справа)
	numberstyle = \tiny,          	 % Размер шрифта для номеров строк
	stepnumber = 1,               	 % Размер шага между двумя номерами строк
	numbersep = 5pt,              	 % Как далеко отстоят номера строк от подсвечиваемого кода
	backgroundcolor = \color{white}, % Цвет фона подсветки - используем \usepackage{color}
	showspaces = false,           	 % Показывать или нет пробелы специальными отступами
	showstringspaces = false,    	 % Показывать или нет пробелы в строках
	showtabs = false,           	 % Показывать или нет табуляцию в строках
	frame = single,              	 % Рисовать рамку вокруг кода
	tabsize = 2,                  	 % Размер табуляции по умолчанию равен 2 пробелам
	captionpos = t,             	 % Позиция заголовка вверху [t] или внизу [b] 
	breaklines = true,           	 % Автоматически переносить строки (да\нет)
	breakatwhitespace = false,   	 % Переносить строки только если есть пробел
	escapeinside = {\%*}{*)}      	 % Если нужно добавить комментарии в коде
}

\begin{document}

\def\contentsname{Содержание}

% Titlepage
\begin{titlepage}
	\begin{center}
		\textsc{Санкт-Петербургский Политехнический 
			Университет Петра Великого\\[5mm]
			Кафедра компьютерных систем и программных технологий}
		
		\vfill
		
		\textbf{Отчёт по лабораторной работе №1\\[3mm]
			Курс: «Администрирование компьютерных сетей»\\[3mm]
			Тема: «Виртуальное макетирование компьютерных сетей»\\[35mm]
			}
	\end{center}
	
	\hfill
	\begin{minipage}{.5\textwidth}
		Выполнил студент:\\[2mm] 
		Ерниязов Тимур Ертлеуевич\\
		Группа: 13541/2\\[5mm]
		
		Проверил:\\[2mm] 
		Малышев Игорь Алексеевич
	\end{minipage}
	\vfill
	\begin{center}
		Санкт-Петербург\\ \the\year\ г.
	\end{center}
\end{titlepage}

% Contents
\tableofcontents
\clearpage
\chapter{Лабораторная работа №1}
\section{Цели работы}
\begin{enumerate}
\item Изучить технологию виртуального макетирования компьютерных сетей в среде VMware Workstation.
\item Разработать и настроить полунатуральный эмулятор компьютерной сети.
\end{enumerate}

\section{Сведения о системе}
Работа производилась на реальной системе, со следующими характеристиками:

\begin{center}
\begin{tabular}{ c c c }
 Элемент & Значение  \\ 
 Имя ОС & Майкрософт Windows 10 Pro (Registered Trademark) \\  
    Версия & 10.0.16299 Сборка 16299 \\     
 RAM & 16 ГБ     \\
  Процессор & Intel(R) Core(TM) i5-7300HQ CPU @ 2.50GHz, 2496 МГц    \\
 
\end{tabular}
\end{center}

Для выполнения работы использовалась \textbf{VMware Workstation 12 pro (12.5.7 build-5813279)}

\section{Создание виртуальных машин}
С помощью средств \textbf{VMware} были созданы виртуальная машины, с использованием ниже представленных операционных систем, с соответствующим выделением оперативной памяти.

\begin{center}
\begin{tabular}{ c c c }
 Название & Версия & Объем RAM  \\ 
NetBSD &7.1.1 64-bit x86&256 MB \\  
    FreeBSD &11.1-RELEASE 64-bit x86&256 MB \\     
 Kali Linux &2017.2 64-bit x86&1.5 GB     \\
  Windows XP Professional &5.1.2600 SP3 Сборка 2600&512 MB   \\
Windows 98 & 4.10.2222A&256 MB    \\
\end{tabular}
\end{center}



\section{Структура сети}
Была создана ККС, состоящая из трех основных сегментов \textbf{(VMnet1, Vmnet2, VMnet3)} и одного вспомогательного \textbf{(VMnet4)}. Каждый представитель подсетей (VMnet1, Vmnet2, VMnet3) имеет один сетевой адаптер, шлюз – два, а маршрутизатор – три сетевых адаптера.


\begin{center}
\begin{tabular}{ c c c c }
 Название сети & Адрес сети & Подключенные узлы  & DHCP \\ 

VMnet1&192.168.40.0&Kali Linux, NetBSD, FreeBSD& -\\ 
VMnet2&192.168.80.0&FreeBSD, Windows XP&+\\ 
VMnet3&192.168.120.0&FreeBSD, Windows 98&-\\ 
VMnet4&192.168.32.0&NetBSD&+\\ 
\end{tabular}
\end{center}


Хост Win98 имеет статический адрес 192.168.120.15, хост Kali Linux также имеет статический адрес 192.168.40.32, а хост WinXP получает адрес 192.168.80.128 динамически с помощью виртуального сервера DHCP.

Маршрутизатору(FreeBSD) были назначены следующие адреса: 192.168.40.2 (для связи с сетью VMnet1), 192.168.80.2 (для связи с сетью VMnet2), 192.168.120.2 (для связи с сетью VMnet3).

Функциональное назначение шлюза (обеспечение взаимодействия ККС с внешними сетями) предполагает наличие какого-нибудь механизма сопряжения IP-адресов. Таким механизмом является служба NAT (преобразование сетевых адресов), подключённая к вспомогательной сети VMnet4, в которую (кроме устройства NAT) входит DHCP-сервер и шлюз. Адрес «внешнего» сетевого адаптера шлюза назначается динамически (DHCP-сервером сети VMnet4) – 192.168.32.128, а адрес «внутреннего» сетевого адаптера шлюза (входящего в сеть VMnet1) статически – 192.168.40.57.


\section{Настройки операционных систем}
\subsection{Windows 98}
В свойствах TCP/IP были заданы:
\begin{enumerate}
\item IP-адрес = 192.168.120.15
\item Маска подсети = 255.255.255.0
\item Шлюз = 192.168.120.2
\end{enumerate}

\subsection{Windows XP}
В свойствах TCP/IP были заданы:
\begin{enumerate}
\item DNS-сервер = 192.168.80.254
\item Шлюз = 192.168.80.2
\end{enumerate}

\subsection{NetBSD}
\begin{enumerate}
\item Узнать названия сетевых адаптеров с помощью команды ifconfig, в моем случае это pcn0, pcn1.
\item Разрешаем ip forwarding добавляя в файл \textbf{/etc/sysctl.conf}:
\begin{lstlisting}[language={}]
net.inet.ip.forwarding=1
\end{lstlisting}
\item Внести следующие настройки в файл \textbf{/etc/rc.conf}:
\begin{enumerate}
\item Указание шлюза по умолчанию:
\begin{lstlisting}[language={}]
defaultroute=192.168.32.2
\end{lstlisting}
\item Задаем ip адрес и сетевую маску для одного из интерфейсов:
\begin{lstlisting}[language={}]
ifconfig_pcn0=inet 192.168.40.57 netmask 255.255.255.0
\end{lstlisting}
\item Разрешаем настройку по DHCP.
\begin{lstlisting}[language={}]
dhclient=yes	
dhclient_flags=pcn1
ifconfig_pcn1=DHCP
\end{lstlisting}
\item Разрешаем запуск NAT:
\begin{lstlisting}[language={}]
ipnat=yes
\end{lstlisting}
\end{enumerate}

\item Задаем правила NAT в файле /etc/ipnat.conf:
\begin{lstlisting}[language={}]
map pcn1 192.168.40.0/24 -> 0/32 portmap tcp/udp 40000:60000
map pcn1 192.168.40.0/24 -> 0/32
\end{lstlisting}
\item В консоли прописываем следующие команды:
\begin{lstlisting}[language={}]
route add -net 192.168.80.0 -netmask 255.255.255.0 192.168.40.2
route add -net 192.168.120.0 -netmask 255.255.255.0 192.168.40.2
\end{lstlisting}
Так как например Windows 98 находится в другом широковещательном домене, были добавлены маршруты, чтобы NetBSD знал куда отвечать.
\end{enumerate}

\subsection{FreeBSD}
\begin{enumerate}
\item Узнать названия сетевых адаптеров с помощью команды ifconfig, в моем случае это em0, em1, em2.
\item Внести следующие настройки в файл \textbf{/etc/rc.conf}:
\begin{lstlisting}[language={}]
gateway_enable="YES"
\end{lstlisting}
\begin{enumerate}
\item Разрешаем ip forwarding при помощи команды:
\begin{lstlisting}[language={}]
defaultrouter=192.168.40.57
\end{lstlisting}
\item Задаем ip адрес и сетевую маску для всех интерфейсов:
\begin{lstlisting}[language={}]
ifconfig_em0=inet 192.168.40.2 netmask 255.255.255.0
ifconfig_em1=inet 192.168.80.2 netmask 255.255.255.0
ifconfig_em2=inet 192.168.120.2 netmask 255.255.255.0
\end{lstlisting}
\item Разрешаем запуск NAT:
\begin{lstlisting}[language={}]
ipnat_enable="YES"
\end{lstlisting}
\end{enumerate}
\item После этого необходимо задать правила NAT для сопряжения адресов. Это делается путем редактирования файла \textbf{/etc/ipnat.rules}:
\begin{lstlisting}[language={}]
map em0 192.168.80.0/24 -> 0.0.0.0/32 portmap tcp/udp 40000:60000 
map em0 192.168.80.0/24 -> 0.0.0.0/32 
map em0 192.168.120.0/24 -> 0.0.0.0/32 portmap tcp/udp 40000:60000
map em0 192.168.120.0/24 -> 0.0.0.0/32
\end{lstlisting}
Эти строки позволяет корректно обрабатывать tcp, udp, icmp пакеты.
\end{enumerate}

\section{Тестирование}
Тестирование заключалось в проверке возможности выхода в интернет из каждой системы, путем отправки ping на адрес 8.8.8.8 (публичный DNS Google). Во всех системах данная команда отработала корректно, что говорит о правильно настроенной ККС.

\section*{Вывод}
В данной работе была рассмотрена эмуляция корпоративной компьютерной сети(ККС), которая содержит три основных и один вспомогательный сегмент сети. Средствами VMWare были созданы:
\begin{itemize}
\item Виртуальные машины, с различными представителями операционных систем.
\item Виртуальные сети(с различными параметрами).
\item Адаптеры для виртуальных машин.
\end{itemize}
Это позволило эмулировать заданную ККС, в которой использовались:
\begin{itemize}
\item Статическая адресация;
\item Динамическое выделение IP адреса;
\item Статическая и динамическая маршрутизация.
\end{itemize}
Также имелась возможность выхода в сеть "Интернет"  из каждой операционной системы.

На мой взгляд \textbf{VMware Workstation} в большей степени подходит для визуализации какой-либо требуемой операционной системы, так как нередки случаи необходимости использования платформозависимого программного обеспечения. Макетирование сетей в данной программе является не лучшим решением, ввиду отсутствия какого-либо визуального редактора. Для подобных целей, лучше использовать специализированное ПО, например \textbf{Graphical Network Simulator 3}.
%------------------------------------------------------------------------------

%\addcontentsline{toc}{section}{Список литературы}
%\bibliography{thesis}
%\bibliographystyle{ugost2008}

\end{document}