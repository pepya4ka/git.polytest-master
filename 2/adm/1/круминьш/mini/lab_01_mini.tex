\documentclass[a4paper, 12pt]{article}		% general format

%%%% Charset
\usepackage{cmap}							% make PDF files searchable and copyable
\usepackage[utf8x]{inputenc} 				% accept different input encodings
\usepackage[english,russian]{babel}   %% загружает пакет многоязыковой вёрстки
\usepackage{fontspec}      %% подготавливает загрузку шрифтов Open Type, True Type и др.
\defaultfontfeatures{Ligatures={TeX},Renderer=Basic}  %% свойства шрифтов по умолчанию
\setmainfont[Ligatures={TeX,Historic}]{Roboto-Light} %% задаёт основной шрифт документа
\setsansfont{Roboto-Light}  

%%%% Graphics
\usepackage[dvipsnames]{xcolor}			% driver-independent color extensions
\usepackage{graphicx}						% enhanced support for graphics
\usepackage{wrapfig}						% produces figures which text can flow around

%%%% Math
\usepackage{amsmath}						% American Mathematical Society (AMS) math facilities
\usepackage{amsfonts}						% fonts from the AMS
\usepackage{amssymb}						% additional math symbols

%%%% Typograpy (don't forget about cm-super)
\usepackage{microtype}						% subliminal refinements towards typographical perfection
\linespread{1.3}							% line spacing
\usepackage[left=2.5cm, right=1.5cm, top=2.5cm, bottom=2.5cm]{geometry}
\setlength{\parindent}{0pt}					% we don't want any paragraph indentation
\usepackage{parskip}						% some distance between paragraphs

%%%% Tables
\usepackage{tabularx}						% tables with variable width columns
\usepackage{multirow}						% for tabularx
\usepackage{hhline}							% for tabularx
\usepackage{tabu}
\usepackage{longtable}

%%%% Graph
\usepackage{tikz}							% package for creating graphics programmatically
\usetikzlibrary{arrows}						% edges for tikz

%%%% Other
\usepackage{url}							% verbatim with URL-sensitive line breaks
\usepackage{fancyvrb}						% sophisticated verbatim text (with box)

\usepackage{listings}
\usepackage{caption}
\DeclareCaptionFont{white}{\color{white}}
\DeclareCaptionFormat{listing}{\colorbox{gray}{\parbox{\dimexpr\textwidth-1.72\fboxsep\relax}{#1#2#3}}}
\captionsetup[lstlisting]{format=listing,labelfont=white,textfont=white,margin=0pt}
\lstset{language=C,
	basicstyle=\footnotesize,
	keepspaces=true,
	tabsize=4,               
	frame=single,                           % Single frame around code
	rulecolor=\color{black},
	captionpos=b,
	showstringspaces=false,	
	abovecaptionskip=-0.9pt,
	xleftmargin=3.4pt,
	xrightmargin=2.6pt,
	breaklines=true,
	postbreak=\raisebox{0ex}[0ex][0ex]{\ensuremath{\color{black}\hookrightarrow\space}},
	xleftmargin=3.2pt,
	literate={а}{{\selectfont\char224}}1
	{~}{{\textasciitilde}}1
	{б}{{\selectfont\char225}}1
	{в}{{\selectfont\char226}}1
	{г}{{\selectfont\char227}}1
	{д}{{\selectfont\char228}}1
	{е}{{\selectfont\char229}}1
	{ё}{{\"e}}1
	{ж}{{\selectfont\char230}}1
	{з}{{\selectfont\char231}}1
	{и}{{\selectfont\char232}}1
	{й}{{\selectfont\char233}}1
	{к}{{\selectfont\char234}}1
	{л}{{\selectfont\char235}}1
	{м}{{\selectfont\char236}}1
	{н}{{\selectfont\char237}}1
	{о}{{\selectfont\char238}}1
	{п}{{\selectfont\char239}}1
	{р}{{\selectfont\char240}}1
	{с}{{\selectfont\char241}}1
	{т}{{\selectfont\char242}}1
	{у}{{\selectfont\char243}}1
	{ф}{{\selectfont\char244}}1
	{х}{{\selectfont\char245}}1
	{ц}{{\selectfont\char246}}1
	{ч}{{\selectfont\char247}}1
	{ш}{{\selectfont\char248}}1
	{щ}{{\selectfont\char249}}1
	{ъ}{{\selectfont\char250}}1
	{ы}{{\selectfont\char251}}1
	{ь}{{\selectfont\char252}}1
	{э}{{\selectfont\char253}}1
	{ю}{{\selectfont\char254}}1
	{я}{{\selectfont\char255}}1
	{А}{{\selectfont\char192}}1
	{Б}{{\selectfont\char193}}1
	{В}{{\selectfont\char194}}1
	{Г}{{\selectfont\char195}}1
	{Д}{{\selectfont\char196}}1
	{Е}{{\selectfont\char197}}1
	{Ё}{{\"E}}1
	{Ж}{{\selectfont\char198}}1
	{З}{{\selectfont\char199}}1
	{И}{{\selectfont\char200}}1
	{Й}{{\selectfont\char201}}1
	{К}{{\selectfont\char202}}1
	{Л}{{\selectfont\char203}}1
	{М}{{\selectfont\char204}}1
	{Н}{{\selectfont\char205}}1
	{О}{{\selectfont\char206}}1
	{П}{{\selectfont\char207}}1
	{Р}{{\selectfont\char208}}1
	{С}{{\selectfont\char209}}1
	{Т}{{\selectfont\char210}}1
	{У}{{\selectfont\char211}}1
	{Ф}{{\selectfont\char212}}1
	{Х}{{\selectfont\char213}}1
	{Ц}{{\selectfont\char214}}1
	{Ч}{{\selectfont\char215}}1
	{Ш}{{\selectfont\char216}}1
	{Щ}{{\selectfont\char217}}1
	{Ъ}{{\selectfont\char218}}1
	{Ы}{{\selectfont\char219}}1
	{Ь}{{\selectfont\char220}}1
	{Э}{{\selectfont\char221}}1
	{Ю}{{\selectfont\char222}}1
	{Я}{{\selectfont\char223}}1,
	extendedchars=true
}

%галочка
\usepackage{amssymb}% http://ctan.org/pkg/amssymb
\usepackage{pifont}% http://ctan.org/pkg/pifont
\newcommand{\cmark}{\ding{52}}%
\newcommand{\xmark}{\ding{56}}
%------------------------------------------------------------------------------
\renewcommand{\labelenumii}{\theenumii}
\renewcommand{\theenumii}{\theenumi.\arabic{enumii}.}
\begin{document}
\begin{titlepage}
\thispagestyle{empty}

\begin{center}
Санкт-Петербургский политехнический университет Петра Великого\\
Институт Информационных Технологий и Управления\\*
Кафедра компьютерных систем и программных технологий\\*
\hrulefill
\end{center}

\vspace{15em}

\begin{center}
\textsc{\textbf{Курсовая работа}}
\vspace{1em}

Дисциплина: \textbf{Методы оптимизации}
\vspace{2em}

Тема: \textbf{Формулировка и решение задачи выбора оптимального решения с использованием различных математических моделей}
\end{center}

\vspace{16em}

\begin{flushleft}
Выполнил студент гр. 53501/3 \hrulefill С.А. Мартынов \\
\vspace{1.5em}
Руководитель, к.т.н.,доц. \hrulefill А.Г. Сиднев\\
\end{flushleft}

\vspace{\fill}

\begin{center}
Санкт-Петербург \\
2015
\end{center}

\end{titlepage}
\setcounter{page}{2}
%\tableofcontents
%\clearpage

%------------------------------------------------------------------------------
%\input{intro}
\section{Цели работы}
\begin{enumerate}
\item Изучить технологию виртуального макетирования компьютерных сетей в среде VMware Workstation.
\item Разработать и настроить полунатуральный эмулятор компьютерной сети.
\end{enumerate}

\section{Сведения о системе}
Работа производилась на реальной системе, со следующими характеристиками:
\tabulinesep = 1mm
\begin{longtabu} to \textwidth {|X[10, c , m ] |X[25, c , m ] | }\firsthline\hline

\textbf{Элемент}&\textbf{Значение}\\ \hline \endfirsthead
	
Имя ОС&Майкрософт Windows 10 Pro (Registered Trademark)\\ \hline
Версия&10.0.16299 Сборка 16299\\ \hline
Установленная оперативная память (RAM) &16,00 ГБ\\ \hline
Процессор&Intel(R) Core(TM) i5-7300HQ CPU @ 2.50GHz, 2496 МГц, ядер: 4, логических процессоров: 4\\ \hline
\end{longtabu}
Для выполнения работы использовалась \textbf{VMware Workstation 12 pro (12.5.7 build-5813279)}

\section{Создание виртуальных машин}
С помощью средств \textbf{VMware} были созданы виртуальная машины, с использованием ниже представленных операционных систем, с соответствующим выделением оперативной памяти.
\tabulinesep = 1mm
\begin{longtabu} to \textwidth {|X[ c , m ] |X[2, c , m ] | X[ c , m ]|}\firsthline\hline
\textbf{Название}&\textbf{Версия}&\textbf{Объем RAM}\\ \hline \endfirsthead
NetBSD&7.1.1 64-bit x86&256 MB\\ \hline
FreeBSD&11.1-RELEASE 64-bit x86&256 MB\\ \hline
Kali Linux&2017.2 64-bit x86&1.5 GB\\ \hline
Windows XP Professional&5.1.2600 SP3 Сборка 2600&512 MB\\ \hline
Windows 98&4.10.2222A&256 MB\\ \hline
\end{longtabu}

\section{Структура сети}
Была создана ККС, состоящая из трех основных сегментов \textbf{(VMnet1, Vmnet2, VMnet3)} и одного вспомогательного \textbf{(VMnet4)}. Каждый представитель подсетей (VMnet1, Vmnet2, VMnet3) имеет один сетевой адаптер, шлюз – два, а маршрутизатор – три сетевых адаптера.

\tabulinesep = 1mm
\begin{longtabu} to \textwidth {|X[ c , m ] |X[ c , m ] | X[2, c , m ]|X[c , m ]|}\firsthline\hline

\textbf{Название сети}&\textbf{Адрес сети}&\textbf{Подключенные узлы}&\textbf{DHCP}\\ \hline \endfirsthead	
VMnet1&192.168.40.0&Kali Linux, NetBSD, FreeBSD&\xmark\\ \hline
VMnet2&192.168.80.0&FreeBSD, Windows XP&\cmark\\ \hline
VMnet3&192.168.120.0&FreeBSD, Windows 98&\xmark\\ \hline
VMnet4&192.168.32.0&NetBSD&\cmark\\ \hline
\end{longtabu}

Хост Win98 имеет статический адрес 192.168.120.15, хост Kali Linux также имеет статический адрес 192.168.40.32, а хост WinXP получает адрес 192.168.80.128 динамически с помощью виртуального сервера DHCP.

Маршрутизатору(FreeBSD) были назначены следующие адреса: 192.168.40.2 (для связи с сетью VMnet1), 192.168.80.2 (для связи с сетью VMnet2), 192.168.120.2 (для связи с сетью VMnet3).

Функциональное назначение шлюза (обеспечение взаимодействия ККС с внешними сетями) предполагает наличие какого-нибудь механизма сопряжения IP-адресов. Таким механизмом является служба NAT (преобразование сетевых адресов), подключённая к вспомогательной сети VMnet4, в которую (кроме устройства NAT) входит DHCP-сервер и шлюз. Адрес «внешнего» сетевого адаптера шлюза назначается динамически (DHCP-сервером сети VMnet4) – 192.168.32.128, а адрес «внутреннего» сетевого адаптера шлюза (входящего в сеть VMnet1) статически – 192.168.40.57.


\section{Настройки операционных систем}
\subsection{Windows 98}
В свойствах TCP/IP были заданы:
\begin{enumerate}
\item IP-адрес = 192.168.120.15
\item Маска подсети = 255.255.255.0
\item Шлюз = 192.168.120.2
\end{enumerate}

\subsection{Windows XP}
В свойствах TCP/IP были заданы:
\begin{enumerate}
\item DNS-сервер = 192.168.80.254
\item Шлюз = 192.168.80.2
\end{enumerate}

\subsection{NetBSD}
\begin{enumerate}
\item Узнать названия сетевых адаптеров с помощью команды ifconfig, в моем случае это pcn0, pcn1.
\item Разрешаем ip forwarding добавляя в файл \textbf{/etc/sysctl.conf}:
\begin{lstlisting}[language={}]
net.inet.ip.forwarding=1
\end{lstlisting}
\item Внести следующие настройки в файл \textbf{/etc/rc.conf}:
\begin{enumerate}
\item Указание шлюза по умолчанию:
\begin{lstlisting}[language={}]
defaultroute=192.168.32.2
\end{lstlisting}
\item Задаем ip адрес и сетевую маску для одного из интерфейсов:
\begin{lstlisting}[language={}]
ifconfig_pcn0=inet 192.168.40.57 netmask 255.255.255.0
\end{lstlisting}
\item Разрешаем настройку по DHCP.
\begin{lstlisting}[language={}]
dhclient=yes	
dhclient_flags=pcn1
ifconfig_pcn1=DHCP
\end{lstlisting}
\item Разрешаем запуск NAT:
\begin{lstlisting}[language={}]
ipnat=yes
\end{lstlisting}
\end{enumerate}

\item Задаем правила NAT в файле /etc/ipnat.conf:
\begin{lstlisting}[language={}]
map pcn1 192.168.40.0/24 -> 0/32 portmap tcp/udp 40000:60000
map pcn1 192.168.40.0/24 -> 0/32
\end{lstlisting}
\item В консоли прописываем следующие команды:
\begin{lstlisting}[language={}]
route add -net 192.168.80.0 -netmask 255.255.255.0 192.168.40.2
route add -net 192.168.120.0 -netmask 255.255.255.0 192.168.40.2
\end{lstlisting}
Так как например Windows 98 находится в другом широковещательном домене, были добавлены маршруты, чтобы NetBSD знал куда отвечать.
\end{enumerate}

\subsection{FreeBSD}
\begin{enumerate}
\item Узнать названия сетевых адаптеров с помощью команды ifconfig, в моем случае это em0, em1, em2.
\item Внести следующие настройки в файл \textbf{/etc/rc.conf}:
\begin{lstlisting}[language={}]
gateway_enable="YES"
\end{lstlisting}
\begin{enumerate}
\item Разрешаем ip forwarding при помощи команды:
\begin{lstlisting}[language={}]
defaultrouter=192.168.40.57
\end{lstlisting}
\item Задаем ip адрес и сетевую маску для всех интерфейсов:
\begin{lstlisting}[language={}]
ifconfig_em0=inet 192.168.40.2 netmask 255.255.255.0
ifconfig_em1=inet 192.168.80.2 netmask 255.255.255.0
ifconfig_em2=inet 192.168.120.2 netmask 255.255.255.0
\end{lstlisting}
\item Разрешаем запуск NAT:
\begin{lstlisting}[language={}]
ipnat_enable="YES"
\end{lstlisting}
\end{enumerate}
\item После этого необходимо задать правила NAT для сопряжения адресов. Это делается путем редактирования файла \textbf{/etc/ipnat.rules}:
\begin{lstlisting}[language={}]
map em0 192.168.80.0/24 -> 0.0.0.0/32 portmap tcp/udp 40000:60000 
map em0 192.168.80.0/24 -> 0.0.0.0/32 
map em0 192.168.120.0/24 -> 0.0.0.0/32 portmap tcp/udp 40000:60000
map em0 192.168.120.0/24 -> 0.0.0.0/32
\end{lstlisting}
Эти строки позволяет корректно обрабатывать tcp, udp, icmp пакеты.
\end{enumerate}

\section{Тестирование}
Тестирование заключалось в проверке возможности выхода в интернет из каждой системы, путем отправки ping на адрес 8.8.8.8 (публичный DNS Google). Во всех системах данная команда отработала корректно, что говорит о правильно настроенной ККС.

\section*{Вывод}
В данной работе была рассмотрена эмуляция корпоративной компьютерной сети(ККС), которая содержит три основных и один вспомогательный сегмент сети. Средствами VMWare были созданы:
\begin{itemize}
\item Виртуальные машины, с различными представителями операционных систем.
\item Виртуальные сети(с различными параметрами).
\item Адаптеры для виртуальных машин.
\end{itemize}
Это позволило эмулировать заданную ККС, в которой использовались:
\begin{itemize}
\item Статическая адресация;
\item Динамическое выделение IP адреса;
\item Статическая и динамическая маршрутизация.
\end{itemize}
Также имелась возможность выхода в сеть "Интернет"  из каждой операционной системы.

На мой взгляд \textbf{VMware Workstation} в большей степени подходит для визуализации какой-либо требуемой операционной системы, так как нередки случаи необходимости использования платформозависимого программного обеспечения. Макетирование сетей в данной программе является не лучшим решением, ввиду отсутствия какого-либо визуального редактора. Для подобных целей, лучше использовать специализированное ПО, например \textbf{Graphical Network Simulator 3}.
%------------------------------------------------------------------------------

%\addcontentsline{toc}{section}{Список литературы}
%\bibliography{thesis}
%\bibliographystyle{ugost2008}

\end{document}